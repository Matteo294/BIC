\documentclass[11pt]{article}

    \usepackage[breakable]{tcolorbox}
    \usepackage{parskip} % Stop auto-indenting (to mimic markdown behaviour)
    
    \usepackage{iftex}
    \ifPDFTeX
    	\usepackage[T1]{fontenc}
    	\usepackage{mathpazo}
    \else
    	\usepackage{fontspec}
    \fi

    % Basic figure setup, for now with no caption control since it's done
    % automatically by Pandoc (which extracts ![](path) syntax from Markdown).
    \usepackage{graphicx}
    % Maintain compatibility with old templates. Remove in nbconvert 6.0
    \let\Oldincludegraphics\includegraphics
    % Ensure that by default, figures have no caption (until we provide a
    % proper Figure object with a Caption API and a way to capture that
    % in the conversion process - todo).
    \usepackage{caption}
    \DeclareCaptionFormat{nocaption}{}
    \captionsetup{format=nocaption,aboveskip=0pt,belowskip=0pt}

    \usepackage{float}
    \floatplacement{figure}{H} % forces figures to be placed at the correct location
    \usepackage{xcolor} % Allow colors to be defined
    \usepackage{enumerate} % Needed for markdown enumerations to work
    \usepackage{geometry} % Used to adjust the document margins
    \usepackage{amsmath} % Equations
    \usepackage{amssymb} % Equations
    \usepackage{textcomp} % defines textquotesingle
    % Hack from http://tex.stackexchange.com/a/47451/13684:
    \AtBeginDocument{%
        \def\PYZsq{\textquotesingle}% Upright quotes in Pygmentized code
    }
    \usepackage{upquote} % Upright quotes for verbatim code
    \usepackage{eurosym} % defines \euro
    \usepackage[mathletters]{ucs} % Extended unicode (utf-8) support
    \usepackage{fancyvrb} % verbatim replacement that allows latex
    \usepackage{grffile} % extends the file name processing of package graphics 
                         % to support a larger range
    \makeatletter % fix for old versions of grffile with XeLaTeX
    \@ifpackagelater{grffile}{2019/11/01}
    {
      % Do nothing on new versions
    }
    {
      \def\Gread@@xetex#1{%
        \IfFileExists{"\Gin@base".bb}%
        {\Gread@eps{\Gin@base.bb}}%
        {\Gread@@xetex@aux#1}%
      }
    }
    \makeatother
    \usepackage[Export]{adjustbox} % Used to constrain images to a maximum size
    \adjustboxset{max size={0.9\linewidth}{0.9\paperheight}}

    % The hyperref package gives us a pdf with properly built
    % internal navigation ('pdf bookmarks' for the table of contents,
    % internal cross-reference links, web links for URLs, etc.)
    \usepackage{hyperref}
    % The default LaTeX title has an obnoxious amount of whitespace. By default,
    % titling removes some of it. It also provides customization options.
    \usepackage{titling}
    \usepackage{longtable} % longtable support required by pandoc >1.10
    \usepackage{booktabs}  % table support for pandoc > 1.12.2
    \usepackage[inline]{enumitem} % IRkernel/repr support (it uses the enumerate* environment)
    \usepackage[normalem]{ulem} % ulem is needed to support strikethroughs (\sout)
                                % normalem makes italics be italics, not underlines
    \usepackage{mathrsfs}
    

    
    % Colors for the hyperref package
    \definecolor{urlcolor}{rgb}{0,.145,.698}
    \definecolor{linkcolor}{rgb}{.71,0.21,0.01}
    \definecolor{citecolor}{rgb}{.12,.54,.11}

    % ANSI colors
    \definecolor{ansi-black}{HTML}{3E424D}
    \definecolor{ansi-black-intense}{HTML}{282C36}
    \definecolor{ansi-red}{HTML}{E75C58}
    \definecolor{ansi-red-intense}{HTML}{B22B31}
    \definecolor{ansi-green}{HTML}{00A250}
    \definecolor{ansi-green-intense}{HTML}{007427}
    \definecolor{ansi-yellow}{HTML}{DDB62B}
    \definecolor{ansi-yellow-intense}{HTML}{B27D12}
    \definecolor{ansi-blue}{HTML}{208FFB}
    \definecolor{ansi-blue-intense}{HTML}{0065CA}
    \definecolor{ansi-magenta}{HTML}{D160C4}
    \definecolor{ansi-magenta-intense}{HTML}{A03196}
    \definecolor{ansi-cyan}{HTML}{60C6C8}
    \definecolor{ansi-cyan-intense}{HTML}{258F8F}
    \definecolor{ansi-white}{HTML}{C5C1B4}
    \definecolor{ansi-white-intense}{HTML}{A1A6B2}
    \definecolor{ansi-default-inverse-fg}{HTML}{FFFFFF}
    \definecolor{ansi-default-inverse-bg}{HTML}{000000}

    % common color for the border for error outputs.
    \definecolor{outerrorbackground}{HTML}{FFDFDF}

    % commands and environments needed by pandoc snippets
    % extracted from the output of `pandoc -s`
    \providecommand{\tightlist}{%
      \setlength{\itemsep}{0pt}\setlength{\parskip}{0pt}}
    \DefineVerbatimEnvironment{Highlighting}{Verbatim}{commandchars=\\\{\}}
    % Add ',fontsize=\small' for more characters per line
    \newenvironment{Shaded}{}{}
    \newcommand{\KeywordTok}[1]{\textcolor[rgb]{0.00,0.44,0.13}{\textbf{{#1}}}}
    \newcommand{\DataTypeTok}[1]{\textcolor[rgb]{0.56,0.13,0.00}{{#1}}}
    \newcommand{\DecValTok}[1]{\textcolor[rgb]{0.25,0.63,0.44}{{#1}}}
    \newcommand{\BaseNTok}[1]{\textcolor[rgb]{0.25,0.63,0.44}{{#1}}}
    \newcommand{\FloatTok}[1]{\textcolor[rgb]{0.25,0.63,0.44}{{#1}}}
    \newcommand{\CharTok}[1]{\textcolor[rgb]{0.25,0.44,0.63}{{#1}}}
    \newcommand{\StringTok}[1]{\textcolor[rgb]{0.25,0.44,0.63}{{#1}}}
    \newcommand{\CommentTok}[1]{\textcolor[rgb]{0.38,0.63,0.69}{\textit{{#1}}}}
    \newcommand{\OtherTok}[1]{\textcolor[rgb]{0.00,0.44,0.13}{{#1}}}
    \newcommand{\AlertTok}[1]{\textcolor[rgb]{1.00,0.00,0.00}{\textbf{{#1}}}}
    \newcommand{\FunctionTok}[1]{\textcolor[rgb]{0.02,0.16,0.49}{{#1}}}
    \newcommand{\RegionMarkerTok}[1]{{#1}}
    \newcommand{\ErrorTok}[1]{\textcolor[rgb]{1.00,0.00,0.00}{\textbf{{#1}}}}
    \newcommand{\NormalTok}[1]{{#1}}
    
    % Additional commands for more recent versions of Pandoc
    \newcommand{\ConstantTok}[1]{\textcolor[rgb]{0.53,0.00,0.00}{{#1}}}
    \newcommand{\SpecialCharTok}[1]{\textcolor[rgb]{0.25,0.44,0.63}{{#1}}}
    \newcommand{\VerbatimStringTok}[1]{\textcolor[rgb]{0.25,0.44,0.63}{{#1}}}
    \newcommand{\SpecialStringTok}[1]{\textcolor[rgb]{0.73,0.40,0.53}{{#1}}}
    \newcommand{\ImportTok}[1]{{#1}}
    \newcommand{\DocumentationTok}[1]{\textcolor[rgb]{0.73,0.13,0.13}{\textit{{#1}}}}
    \newcommand{\AnnotationTok}[1]{\textcolor[rgb]{0.38,0.63,0.69}{\textbf{\textit{{#1}}}}}
    \newcommand{\CommentVarTok}[1]{\textcolor[rgb]{0.38,0.63,0.69}{\textbf{\textit{{#1}}}}}
    \newcommand{\VariableTok}[1]{\textcolor[rgb]{0.10,0.09,0.49}{{#1}}}
    \newcommand{\ControlFlowTok}[1]{\textcolor[rgb]{0.00,0.44,0.13}{\textbf{{#1}}}}
    \newcommand{\OperatorTok}[1]{\textcolor[rgb]{0.40,0.40,0.40}{{#1}}}
    \newcommand{\BuiltInTok}[1]{{#1}}
    \newcommand{\ExtensionTok}[1]{{#1}}
    \newcommand{\PreprocessorTok}[1]{\textcolor[rgb]{0.74,0.48,0.00}{{#1}}}
    \newcommand{\AttributeTok}[1]{\textcolor[rgb]{0.49,0.56,0.16}{{#1}}}
    \newcommand{\InformationTok}[1]{\textcolor[rgb]{0.38,0.63,0.69}{\textbf{\textit{{#1}}}}}
    \newcommand{\WarningTok}[1]{\textcolor[rgb]{0.38,0.63,0.69}{\textbf{\textit{{#1}}}}}
    
    
    % Define a nice break command that doesn't care if a line doesn't already
    % exist.
    \def\br{\hspace*{\fill} \\* }
    % Math Jax compatibility definitions
    \def\gt{>}
    \def\lt{<}
    \let\Oldtex\TeX
    \let\Oldlatex\LaTeX
    \renewcommand{\TeX}{\textrm{\Oldtex}}
    \renewcommand{\LaTeX}{\textrm{\Oldlatex}}
    % Document parameters
    % Document title
    \title{Results\_03}
    
    
    
    
    
% Pygments definitions
\makeatletter
\def\PY@reset{\let\PY@it=\relax \let\PY@bf=\relax%
    \let\PY@ul=\relax \let\PY@tc=\relax%
    \let\PY@bc=\relax \let\PY@ff=\relax}
\def\PY@tok#1{\csname PY@tok@#1\endcsname}
\def\PY@toks#1+{\ifx\relax#1\empty\else%
    \PY@tok{#1}\expandafter\PY@toks\fi}
\def\PY@do#1{\PY@bc{\PY@tc{\PY@ul{%
    \PY@it{\PY@bf{\PY@ff{#1}}}}}}}
\def\PY#1#2{\PY@reset\PY@toks#1+\relax+\PY@do{#2}}

\@namedef{PY@tok@w}{\def\PY@tc##1{\textcolor[rgb]{0.73,0.73,0.73}{##1}}}
\@namedef{PY@tok@c}{\let\PY@it=\textit\def\PY@tc##1{\textcolor[rgb]{0.25,0.50,0.50}{##1}}}
\@namedef{PY@tok@cp}{\def\PY@tc##1{\textcolor[rgb]{0.74,0.48,0.00}{##1}}}
\@namedef{PY@tok@k}{\let\PY@bf=\textbf\def\PY@tc##1{\textcolor[rgb]{0.00,0.50,0.00}{##1}}}
\@namedef{PY@tok@kp}{\def\PY@tc##1{\textcolor[rgb]{0.00,0.50,0.00}{##1}}}
\@namedef{PY@tok@kt}{\def\PY@tc##1{\textcolor[rgb]{0.69,0.00,0.25}{##1}}}
\@namedef{PY@tok@o}{\def\PY@tc##1{\textcolor[rgb]{0.40,0.40,0.40}{##1}}}
\@namedef{PY@tok@ow}{\let\PY@bf=\textbf\def\PY@tc##1{\textcolor[rgb]{0.67,0.13,1.00}{##1}}}
\@namedef{PY@tok@nb}{\def\PY@tc##1{\textcolor[rgb]{0.00,0.50,0.00}{##1}}}
\@namedef{PY@tok@nf}{\def\PY@tc##1{\textcolor[rgb]{0.00,0.00,1.00}{##1}}}
\@namedef{PY@tok@nc}{\let\PY@bf=\textbf\def\PY@tc##1{\textcolor[rgb]{0.00,0.00,1.00}{##1}}}
\@namedef{PY@tok@nn}{\let\PY@bf=\textbf\def\PY@tc##1{\textcolor[rgb]{0.00,0.00,1.00}{##1}}}
\@namedef{PY@tok@ne}{\let\PY@bf=\textbf\def\PY@tc##1{\textcolor[rgb]{0.82,0.25,0.23}{##1}}}
\@namedef{PY@tok@nv}{\def\PY@tc##1{\textcolor[rgb]{0.10,0.09,0.49}{##1}}}
\@namedef{PY@tok@no}{\def\PY@tc##1{\textcolor[rgb]{0.53,0.00,0.00}{##1}}}
\@namedef{PY@tok@nl}{\def\PY@tc##1{\textcolor[rgb]{0.63,0.63,0.00}{##1}}}
\@namedef{PY@tok@ni}{\let\PY@bf=\textbf\def\PY@tc##1{\textcolor[rgb]{0.60,0.60,0.60}{##1}}}
\@namedef{PY@tok@na}{\def\PY@tc##1{\textcolor[rgb]{0.49,0.56,0.16}{##1}}}
\@namedef{PY@tok@nt}{\let\PY@bf=\textbf\def\PY@tc##1{\textcolor[rgb]{0.00,0.50,0.00}{##1}}}
\@namedef{PY@tok@nd}{\def\PY@tc##1{\textcolor[rgb]{0.67,0.13,1.00}{##1}}}
\@namedef{PY@tok@s}{\def\PY@tc##1{\textcolor[rgb]{0.73,0.13,0.13}{##1}}}
\@namedef{PY@tok@sd}{\let\PY@it=\textit\def\PY@tc##1{\textcolor[rgb]{0.73,0.13,0.13}{##1}}}
\@namedef{PY@tok@si}{\let\PY@bf=\textbf\def\PY@tc##1{\textcolor[rgb]{0.73,0.40,0.53}{##1}}}
\@namedef{PY@tok@se}{\let\PY@bf=\textbf\def\PY@tc##1{\textcolor[rgb]{0.73,0.40,0.13}{##1}}}
\@namedef{PY@tok@sr}{\def\PY@tc##1{\textcolor[rgb]{0.73,0.40,0.53}{##1}}}
\@namedef{PY@tok@ss}{\def\PY@tc##1{\textcolor[rgb]{0.10,0.09,0.49}{##1}}}
\@namedef{PY@tok@sx}{\def\PY@tc##1{\textcolor[rgb]{0.00,0.50,0.00}{##1}}}
\@namedef{PY@tok@m}{\def\PY@tc##1{\textcolor[rgb]{0.40,0.40,0.40}{##1}}}
\@namedef{PY@tok@gh}{\let\PY@bf=\textbf\def\PY@tc##1{\textcolor[rgb]{0.00,0.00,0.50}{##1}}}
\@namedef{PY@tok@gu}{\let\PY@bf=\textbf\def\PY@tc##1{\textcolor[rgb]{0.50,0.00,0.50}{##1}}}
\@namedef{PY@tok@gd}{\def\PY@tc##1{\textcolor[rgb]{0.63,0.00,0.00}{##1}}}
\@namedef{PY@tok@gi}{\def\PY@tc##1{\textcolor[rgb]{0.00,0.63,0.00}{##1}}}
\@namedef{PY@tok@gr}{\def\PY@tc##1{\textcolor[rgb]{1.00,0.00,0.00}{##1}}}
\@namedef{PY@tok@ge}{\let\PY@it=\textit}
\@namedef{PY@tok@gs}{\let\PY@bf=\textbf}
\@namedef{PY@tok@gp}{\let\PY@bf=\textbf\def\PY@tc##1{\textcolor[rgb]{0.00,0.00,0.50}{##1}}}
\@namedef{PY@tok@go}{\def\PY@tc##1{\textcolor[rgb]{0.53,0.53,0.53}{##1}}}
\@namedef{PY@tok@gt}{\def\PY@tc##1{\textcolor[rgb]{0.00,0.27,0.87}{##1}}}
\@namedef{PY@tok@err}{\def\PY@bc##1{{\setlength{\fboxsep}{\string -\fboxrule}\fcolorbox[rgb]{1.00,0.00,0.00}{1,1,1}{\strut ##1}}}}
\@namedef{PY@tok@kc}{\let\PY@bf=\textbf\def\PY@tc##1{\textcolor[rgb]{0.00,0.50,0.00}{##1}}}
\@namedef{PY@tok@kd}{\let\PY@bf=\textbf\def\PY@tc##1{\textcolor[rgb]{0.00,0.50,0.00}{##1}}}
\@namedef{PY@tok@kn}{\let\PY@bf=\textbf\def\PY@tc##1{\textcolor[rgb]{0.00,0.50,0.00}{##1}}}
\@namedef{PY@tok@kr}{\let\PY@bf=\textbf\def\PY@tc##1{\textcolor[rgb]{0.00,0.50,0.00}{##1}}}
\@namedef{PY@tok@bp}{\def\PY@tc##1{\textcolor[rgb]{0.00,0.50,0.00}{##1}}}
\@namedef{PY@tok@fm}{\def\PY@tc##1{\textcolor[rgb]{0.00,0.00,1.00}{##1}}}
\@namedef{PY@tok@vc}{\def\PY@tc##1{\textcolor[rgb]{0.10,0.09,0.49}{##1}}}
\@namedef{PY@tok@vg}{\def\PY@tc##1{\textcolor[rgb]{0.10,0.09,0.49}{##1}}}
\@namedef{PY@tok@vi}{\def\PY@tc##1{\textcolor[rgb]{0.10,0.09,0.49}{##1}}}
\@namedef{PY@tok@vm}{\def\PY@tc##1{\textcolor[rgb]{0.10,0.09,0.49}{##1}}}
\@namedef{PY@tok@sa}{\def\PY@tc##1{\textcolor[rgb]{0.73,0.13,0.13}{##1}}}
\@namedef{PY@tok@sb}{\def\PY@tc##1{\textcolor[rgb]{0.73,0.13,0.13}{##1}}}
\@namedef{PY@tok@sc}{\def\PY@tc##1{\textcolor[rgb]{0.73,0.13,0.13}{##1}}}
\@namedef{PY@tok@dl}{\def\PY@tc##1{\textcolor[rgb]{0.73,0.13,0.13}{##1}}}
\@namedef{PY@tok@s2}{\def\PY@tc##1{\textcolor[rgb]{0.73,0.13,0.13}{##1}}}
\@namedef{PY@tok@sh}{\def\PY@tc##1{\textcolor[rgb]{0.73,0.13,0.13}{##1}}}
\@namedef{PY@tok@s1}{\def\PY@tc##1{\textcolor[rgb]{0.73,0.13,0.13}{##1}}}
\@namedef{PY@tok@mb}{\def\PY@tc##1{\textcolor[rgb]{0.40,0.40,0.40}{##1}}}
\@namedef{PY@tok@mf}{\def\PY@tc##1{\textcolor[rgb]{0.40,0.40,0.40}{##1}}}
\@namedef{PY@tok@mh}{\def\PY@tc##1{\textcolor[rgb]{0.40,0.40,0.40}{##1}}}
\@namedef{PY@tok@mi}{\def\PY@tc##1{\textcolor[rgb]{0.40,0.40,0.40}{##1}}}
\@namedef{PY@tok@il}{\def\PY@tc##1{\textcolor[rgb]{0.40,0.40,0.40}{##1}}}
\@namedef{PY@tok@mo}{\def\PY@tc##1{\textcolor[rgb]{0.40,0.40,0.40}{##1}}}
\@namedef{PY@tok@ch}{\let\PY@it=\textit\def\PY@tc##1{\textcolor[rgb]{0.25,0.50,0.50}{##1}}}
\@namedef{PY@tok@cm}{\let\PY@it=\textit\def\PY@tc##1{\textcolor[rgb]{0.25,0.50,0.50}{##1}}}
\@namedef{PY@tok@cpf}{\let\PY@it=\textit\def\PY@tc##1{\textcolor[rgb]{0.25,0.50,0.50}{##1}}}
\@namedef{PY@tok@c1}{\let\PY@it=\textit\def\PY@tc##1{\textcolor[rgb]{0.25,0.50,0.50}{##1}}}
\@namedef{PY@tok@cs}{\let\PY@it=\textit\def\PY@tc##1{\textcolor[rgb]{0.25,0.50,0.50}{##1}}}

\def\PYZbs{\char`\\}
\def\PYZus{\char`\_}
\def\PYZob{\char`\{}
\def\PYZcb{\char`\}}
\def\PYZca{\char`\^}
\def\PYZam{\char`\&}
\def\PYZlt{\char`\<}
\def\PYZgt{\char`\>}
\def\PYZsh{\char`\#}
\def\PYZpc{\char`\%}
\def\PYZdl{\char`\$}
\def\PYZhy{\char`\-}
\def\PYZsq{\char`\'}
\def\PYZdq{\char`\"}
\def\PYZti{\char`\~}
% for compatibility with earlier versions
\def\PYZat{@}
\def\PYZlb{[}
\def\PYZrb{]}
\makeatother


    % For linebreaks inside Verbatim environment from package fancyvrb. 
    \makeatletter
        \newbox\Wrappedcontinuationbox 
        \newbox\Wrappedvisiblespacebox 
        \newcommand*\Wrappedvisiblespace {\textcolor{red}{\textvisiblespace}} 
        \newcommand*\Wrappedcontinuationsymbol {\textcolor{red}{\llap{\tiny$\m@th\hookrightarrow$}}} 
        \newcommand*\Wrappedcontinuationindent {3ex } 
        \newcommand*\Wrappedafterbreak {\kern\Wrappedcontinuationindent\copy\Wrappedcontinuationbox} 
        % Take advantage of the already applied Pygments mark-up to insert 
        % potential linebreaks for TeX processing. 
        %        {, <, #, %, $, ' and ": go to next line. 
        %        _, }, ^, &, >, - and ~: stay at end of broken line. 
        % Use of \textquotesingle for straight quote. 
        \newcommand*\Wrappedbreaksatspecials {% 
            \def\PYGZus{\discretionary{\char`\_}{\Wrappedafterbreak}{\char`\_}}% 
            \def\PYGZob{\discretionary{}{\Wrappedafterbreak\char`\{}{\char`\{}}% 
            \def\PYGZcb{\discretionary{\char`\}}{\Wrappedafterbreak}{\char`\}}}% 
            \def\PYGZca{\discretionary{\char`\^}{\Wrappedafterbreak}{\char`\^}}% 
            \def\PYGZam{\discretionary{\char`\&}{\Wrappedafterbreak}{\char`\&}}% 
            \def\PYGZlt{\discretionary{}{\Wrappedafterbreak\char`\<}{\char`\<}}% 
            \def\PYGZgt{\discretionary{\char`\>}{\Wrappedafterbreak}{\char`\>}}% 
            \def\PYGZsh{\discretionary{}{\Wrappedafterbreak\char`\#}{\char`\#}}% 
            \def\PYGZpc{\discretionary{}{\Wrappedafterbreak\char`\%}{\char`\%}}% 
            \def\PYGZdl{\discretionary{}{\Wrappedafterbreak\char`\$}{\char`\$}}% 
            \def\PYGZhy{\discretionary{\char`\-}{\Wrappedafterbreak}{\char`\-}}% 
            \def\PYGZsq{\discretionary{}{\Wrappedafterbreak\textquotesingle}{\textquotesingle}}% 
            \def\PYGZdq{\discretionary{}{\Wrappedafterbreak\char`\"}{\char`\"}}% 
            \def\PYGZti{\discretionary{\char`\~}{\Wrappedafterbreak}{\char`\~}}% 
        } 
        % Some characters . , ; ? ! / are not pygmentized. 
        % This macro makes them "active" and they will insert potential linebreaks 
        \newcommand*\Wrappedbreaksatpunct {% 
            \lccode`\~`\.\lowercase{\def~}{\discretionary{\hbox{\char`\.}}{\Wrappedafterbreak}{\hbox{\char`\.}}}% 
            \lccode`\~`\,\lowercase{\def~}{\discretionary{\hbox{\char`\,}}{\Wrappedafterbreak}{\hbox{\char`\,}}}% 
            \lccode`\~`\;\lowercase{\def~}{\discretionary{\hbox{\char`\;}}{\Wrappedafterbreak}{\hbox{\char`\;}}}% 
            \lccode`\~`\:\lowercase{\def~}{\discretionary{\hbox{\char`\:}}{\Wrappedafterbreak}{\hbox{\char`\:}}}% 
            \lccode`\~`\?\lowercase{\def~}{\discretionary{\hbox{\char`\?}}{\Wrappedafterbreak}{\hbox{\char`\?}}}% 
            \lccode`\~`\!\lowercase{\def~}{\discretionary{\hbox{\char`\!}}{\Wrappedafterbreak}{\hbox{\char`\!}}}% 
            \lccode`\~`\/\lowercase{\def~}{\discretionary{\hbox{\char`\/}}{\Wrappedafterbreak}{\hbox{\char`\/}}}% 
            \catcode`\.\active
            \catcode`\,\active 
            \catcode`\;\active
            \catcode`\:\active
            \catcode`\?\active
            \catcode`\!\active
            \catcode`\/\active 
            \lccode`\~`\~ 	
        }
    \makeatother

    \let\OriginalVerbatim=\Verbatim
    \makeatletter
    \renewcommand{\Verbatim}[1][1]{%
        %\parskip\z@skip
        \sbox\Wrappedcontinuationbox {\Wrappedcontinuationsymbol}%
        \sbox\Wrappedvisiblespacebox {\FV@SetupFont\Wrappedvisiblespace}%
        \def\FancyVerbFormatLine ##1{\hsize\linewidth
            \vtop{\raggedright\hyphenpenalty\z@\exhyphenpenalty\z@
                \doublehyphendemerits\z@\finalhyphendemerits\z@
                \strut ##1\strut}%
        }%
        % If the linebreak is at a space, the latter will be displayed as visible
        % space at end of first line, and a continuation symbol starts next line.
        % Stretch/shrink are however usually zero for typewriter font.
        \def\FV@Space {%
            \nobreak\hskip\z@ plus\fontdimen3\font minus\fontdimen4\font
            \discretionary{\copy\Wrappedvisiblespacebox}{\Wrappedafterbreak}
            {\kern\fontdimen2\font}%
        }%
        
        % Allow breaks at special characters using \PYG... macros.
        \Wrappedbreaksatspecials
        % Breaks at punctuation characters . , ; ? ! and / need catcode=\active 	
        \OriginalVerbatim[#1,codes*=\Wrappedbreaksatpunct]%
    }
    \makeatother

    % Exact colors from NB
    \definecolor{incolor}{HTML}{303F9F}
    \definecolor{outcolor}{HTML}{D84315}
    \definecolor{cellborder}{HTML}{CFCFCF}
    \definecolor{cellbackground}{HTML}{F7F7F7}
    
    % prompt
    \makeatletter
    \newcommand{\boxspacing}{\kern\kvtcb@left@rule\kern\kvtcb@boxsep}
    \makeatother
    \newcommand{\prompt}[4]{
        {\ttfamily\llap{{\color{#2}[#3]:\hspace{3pt}#4}}\vspace{-\baselineskip}}
    }
    

    
    % Prevent overflowing lines due to hard-to-break entities
    \sloppy 
    % Setup hyperref package
    \hypersetup{
      breaklinks=true,  % so long urls are correctly broken across lines
      colorlinks=true,
      urlcolor=urlcolor,
      linkcolor=linkcolor,
      citecolor=citecolor,
      }
    % Slightly bigger margins than the latex defaults
    
    \geometry{verbose,tmargin=1in,bmargin=1in,lmargin=1in,rmargin=1in}
    
    

\begin{document}
    
    \maketitle
    
    

    
    \section*{Matteo Zortea, Alessandro Rizzi, Marvin
Wolf}\label{matteo-zortea-alessandro-rizzi-and-marvin-wolf}

\subsection*{Brain Inspired Computing - Sheet
2}\label{brain-inspired-computing---sheet-2}

\subsection*{Group G3A8}\label{group-g3a8}

\subsubsection*{Exercises solved}\label{exercises-solved}

Ex 1: a) b) c)\\Ex 2: a) b)\\Ex 3: 1) 2) 3) 4)\\Ex 4: a) b) c) d)

    \section*{Exercise 1}\label{exercise-1}

The LIF model is described by a differential equation of the type

\begin{equation}
    C_m \, \frac{du(t)}{dt} = g_L \, \left(E_L - u(t)\right) + I_{ext}(t)
\end{equation}

when the membrane potential is below the treshold value $u_{th}$.\\If
one then wants to take account of various effects such as inhibitory or
excitatory pre-synaptic inputs, spike rate adaption or refractary
periods, one can add more terms to equation (1), but the equation can
always be recasted into one that is formally analogue to (1), as we will
show in exercise 2. Hence we can solve equation (1) having in mind that
more complex equations have the same solution, formally speaking.\\It is
convenient to work in the Laplace's domain. The differential equation
becomes

\begin{equation*}
    s \tilde u(s) - u(0) = \frac{E_L}{s\tau} - \frac{\tilde u(s)}{\tau} + \frac{\tilde I_{ext}(s)}{C_m}
\end{equation*}

where we have defined $\tau \equiv C_m/g_L$. The last equation can be
rearranged into

\begin{equation*}
    \tilde u(s) = \frac{u(0)}{s+1/\tau} + \frac{E_L}{s \tau (s+1/\tau)} + \frac{\tilde I(s)}{C_m(s+1/\tau)} = \frac{u(0)}{s+1/\tau} +  E_L \left(\frac{1}{s} - \frac{1}{s + 1/\tau} \right)+ \frac{\tilde I(s)}{C_m(s+1/\tau)}
\end{equation*}

    We can now proceed by antitransforming \[
    u(t) = u(t=0) e^{-t/\tau} + E_L \left(1 - e^{-t/\tau}\right) + \mathcal{L}^{-1} \biggl  [\frac{\tilde I(s)}{C_m(s+1/\tau)}\biggr].
\]

    We have now two ways of solving the equation. One is to calculate the
transform of the input current, pop it in the last formula and then
antitransform the last term. The other one consists in using directly
the time expression of the input current and make use of the convolution
property of the Laplace transform
$\mathcal{L} [f \star g] = \mathcal{L}[f] \cdot \mathcal{L}[g]$ which
yields

\begin{equation*}
    u(t) = u(t=0) \, e^{-t/\tau} + E_L \left(1 - e^{-t/\tau}\right) + \frac{1}{C_m} \int_0^t I(t') \, e^{-(t-t')/\tau} \, dt'
\end{equation*}

    \subsubsection*{a)}\label{a}

We use the temporal expression of the current and we get directly

\begin{equation}
    u(t) = u(t=0) \, e^{-t/\tau} + E_L \left(1 - e^{-t/\tau}\right) + \frac{I_0}{g_L} e^{-(t-t_0)/\tau}
\end{equation}

    \subsubsection*{b)}\label{b}

\begin{gather*}
    u(t) = u(t=0) \, e^{-t/\tau} + E_L \left(1 - e^{-t/\tau}\right) + \frac{I_0}{C_m} \int_{0}^t I(t') \, e^{-(t-t')/\tau} \ \Theta(t'-t_0) \, dt' = \\ 
    = u(t=0) \, e^{-t/\tau} + E_L \left(1 - e^{-t/\tau}\right) + \frac{I_0}{C_m} \Theta(t-t_0) \ \int_{t_0}^t I(t') \, e^{-(t-t')/\tau} \, dt' = \\
    = u(t=0) \, e^{-t/\tau} + E_L \left(1 - e^{-t/\tau}\right) + \frac{I_0}{C_m} \Theta(t-t_0) \left(1 - e^{-(t-t_0)/\tau}\right)
\end{gather*}

    \subsubsection*{c)}\label{c}

It is clear from the convolution term that the result is an
exponentially growing term with constant $\tau$ that gets counteracted
by another exponential term with time constant $\tau_s$. Indeed

\begin{gather*}
    u(t) = u(t=0) \, e^{-t/\tau} + E_L \left(1 - e^{-t/\tau}\right) + \frac{I_0}{\tau_s g_L} \Theta(t-t_0) \int_{t_0}^t \exp\left(\frac{t_0-t'}{\tau_s} + \frac{t'-t}{\tau}\right) \, dt' = \\
    = u(t=0) \, e^{-t/\tau} + E_L \left(1 - e^{-t/\tau}\right) + \frac{I_0}{\tau_s g_L} \Theta(t-t_0) \int_{t_0}^t \exp\left(t'(1/\tau-1/\tau_s)\right) \, \exp\left(t_0/\tau_s - t/\tau\right) \, dt' = \\
    = u(t=0) \, e^{-t/\tau} + E_L \left(1 - e^{-t/\tau}\right) + \frac{I_0}{g_L} \frac{\tau}{\tau_s - \tau} \, \Theta(t-t_0) \, \left[\exp\left(-(t-t_0)/\tau_s\right) - \exp\left(-(t-t_0)/\tau\right)\right]
\end{gather*}

If $\tau_s \to +\infty$ (i.e. the counterating terms never enters in
action) we obtain result b).\\The qualitative plots follow (arbitatry
units)

    \begin{tcolorbox}[breakable, size=fbox, boxrule=1pt, pad at break*=1mm,colback=cellbackground, colframe=cellborder]
\prompt{In}{incolor}{1}{\boxspacing}
\begin{Verbatim}[commandchars=\\\{\}]
\PY{k+kn}{import} \PY{n+nn}{numpy} \PY{k}{as} \PY{n+nn}{np} 
\PY{k+kn}{from} \PY{n+nn}{matplotlib} \PY{k+kn}{import} \PY{n}{pyplot} \PY{k}{as} \PY{n}{plt} 
\PY{o}{\PYZpc{}}\PY{k}{matplotlib} inline

\PY{n}{EL} \PY{o}{=} \PY{l+m+mi}{0}
\PY{n}{u0} \PY{o}{=} \PY{n}{EL}
\PY{n}{tau} \PY{o}{=} \PY{l+m+mi}{1}
\PY{n}{t0} \PY{o}{=} \PY{l+m+mi}{2}
\PY{n}{tau\PYZus{}s} \PY{o}{=} \PY{l+m+mf}{0.5}
\PY{n}{tau\PYZus{}eff} \PY{o}{=} \PY{n}{tau\PYZus{}s}\PY{o}{*}\PY{n}{tau}\PY{o}{/}\PY{p}{(}\PY{n}{tau\PYZus{}s}\PY{o}{\PYZhy{}}\PY{n}{tau}\PY{p}{)}

\PY{n}{t} \PY{o}{=} \PY{n}{np}\PY{o}{.}\PY{n}{linspace}\PY{p}{(}\PY{l+m+mi}{0}\PY{p}{,}\PY{l+m+mi}{10}\PY{p}{,}\PY{l+m+mi}{1000}\PY{p}{)}

\PY{k}{def} \PY{n+nf}{f0}\PY{p}{(}\PY{n}{t}\PY{p}{)}\PY{p}{:}
    \PY{k}{return} \PY{n}{u0}\PY{o}{*}\PY{n}{np}\PY{o}{.}\PY{n}{exp}\PY{p}{(}\PY{o}{\PYZhy{}}\PY{n}{t}\PY{o}{/}\PY{n}{tau}\PY{p}{)} \PY{o}{+} \PY{n}{EL}\PY{o}{*}\PY{p}{(}\PY{l+m+mi}{1}\PY{o}{\PYZhy{}}\PY{n}{np}\PY{o}{.}\PY{n}{exp}\PY{p}{(}\PY{o}{\PYZhy{}}\PY{n}{t}\PY{o}{/}\PY{n}{tau}\PY{p}{)}\PY{p}{)}

\PY{k}{def} \PY{n+nf}{f1}\PY{p}{(}\PY{n}{t}\PY{p}{)}\PY{p}{:}
    \PY{n}{ret} \PY{o}{=} \PY{p}{[}\PY{p}{]}
    \PY{k}{for} \PY{n}{tt} \PY{o+ow}{in} \PY{n}{t}\PY{p}{:}
        \PY{k}{if} \PY{n}{tt} \PY{o}{\PYZlt{}} \PY{n}{t0}\PY{p}{:}
            \PY{n}{ret}\PY{o}{.}\PY{n}{append}\PY{p}{(}\PY{n}{f0}\PY{p}{(}\PY{n}{tt}\PY{p}{)}\PY{p}{)}
        \PY{k}{else}\PY{p}{:}
            \PY{n}{ret}\PY{o}{.}\PY{n}{append}\PY{p}{(}\PY{n}{f0}\PY{p}{(}\PY{n}{tt}\PY{p}{)} \PY{o}{+} \PY{n}{np}\PY{o}{.}\PY{n}{exp}\PY{p}{(}\PY{o}{\PYZhy{}}\PY{p}{(}\PY{n}{tt}\PY{o}{\PYZhy{}}\PY{n}{t0}\PY{p}{)}\PY{o}{/}\PY{n}{tau}\PY{p}{)}\PY{p}{)}
    \PY{k}{return} \PY{n}{ret}

\PY{k}{def} \PY{n+nf}{f2}\PY{p}{(}\PY{n}{t}\PY{p}{)}\PY{p}{:}
    \PY{n}{ret} \PY{o}{=} \PY{p}{[}\PY{p}{]}
    \PY{k}{for} \PY{n}{tt} \PY{o+ow}{in} \PY{n}{t}\PY{p}{:}
        \PY{k}{if} \PY{n}{tt} \PY{o}{\PYZlt{}} \PY{n}{t0}\PY{p}{:}
            \PY{n}{ret}\PY{o}{.}\PY{n}{append}\PY{p}{(}\PY{n}{f0}\PY{p}{(}\PY{n}{tt}\PY{p}{)}\PY{p}{)}
        \PY{k}{else}\PY{p}{:}
            \PY{n}{ret}\PY{o}{.}\PY{n}{append}\PY{p}{(}\PY{n}{f0}\PY{p}{(}\PY{n}{tt}\PY{p}{)} \PY{o}{+} \PY{p}{(}\PY{l+m+mi}{1} \PY{o}{\PYZhy{}} \PY{n}{np}\PY{o}{.}\PY{n}{exp}\PY{p}{(}\PY{o}{\PYZhy{}}\PY{p}{(}\PY{n}{tt}\PY{o}{\PYZhy{}}\PY{n}{t0}\PY{p}{)}\PY{o}{/}\PY{n}{tau}\PY{p}{)}\PY{p}{)}\PY{p}{)}
    \PY{k}{return} \PY{n}{ret}

\PY{k}{def} \PY{n+nf}{f3}\PY{p}{(}\PY{n}{t}\PY{p}{)}\PY{p}{:}
    \PY{n}{ret} \PY{o}{=} \PY{p}{[}\PY{p}{]}
    \PY{k}{for} \PY{n}{tt} \PY{o+ow}{in} \PY{n}{t}\PY{p}{:}
        \PY{k}{if} \PY{n}{tt} \PY{o}{\PYZlt{}} \PY{n}{t0}\PY{p}{:}
            \PY{n}{ret}\PY{o}{.}\PY{n}{append}\PY{p}{(}\PY{n}{f0}\PY{p}{(}\PY{n}{tt}\PY{p}{)}\PY{p}{)}
        \PY{k}{else}\PY{p}{:}
            \PY{n}{ret}\PY{o}{.}\PY{n}{append}\PY{p}{(}\PY{n}{f0}\PY{p}{(}\PY{n}{tt}\PY{p}{)} \PY{o}{+} \PY{n}{tau\PYZus{}s}\PY{o}{/}\PY{p}{(}\PY{n}{tau\PYZus{}s}\PY{o}{\PYZhy{}}\PY{n}{tau}\PY{p}{)} \PY{o}{*} \PY{p}{(}\PY{n}{np}\PY{o}{.}\PY{n}{exp}\PY{p}{(}\PY{o}{\PYZhy{}}\PY{p}{(}\PY{n}{tt}\PY{o}{\PYZhy{}}\PY{n}{t0}\PY{p}{)}\PY{o}{/}\PY{n}{tau\PYZus{}s}\PY{p}{)} \PY{o}{\PYZhy{}} \PY{n}{np}\PY{o}{.}\PY{n}{exp}\PY{p}{(}\PY{o}{\PYZhy{}}\PY{p}{(}\PY{n}{tt}\PY{o}{\PYZhy{}}\PY{n}{t0}\PY{p}{)}\PY{o}{/}\PY{n}{tau}\PY{p}{)}\PY{p}{)}\PY{p}{)}
    \PY{k}{return} \PY{n}{ret}

\PY{n}{plt}\PY{o}{.}\PY{n}{plot}\PY{p}{(}\PY{n}{t}\PY{p}{,} \PY{n}{f0}\PY{p}{(}\PY{n}{t}\PY{p}{)}\PY{p}{)}
\PY{n}{plt}\PY{o}{.}\PY{n}{plot}\PY{p}{(}\PY{n}{t}\PY{p}{,} \PY{n}{f1}\PY{p}{(}\PY{n}{t}\PY{p}{)}\PY{p}{,} \PY{n}{label}\PY{o}{=}\PY{l+s+sa}{r}\PY{l+s+s1}{\PYZsq{}}\PY{l+s+s1}{input \PYZdl{}I\PYZus{}1(t)\PYZdl{}}\PY{l+s+s1}{\PYZsq{}}\PY{p}{)}
\PY{n}{plt}\PY{o}{.}\PY{n}{plot}\PY{p}{(}\PY{n}{t}\PY{p}{,} \PY{n}{f2}\PY{p}{(}\PY{n}{t}\PY{p}{)}\PY{p}{,} \PY{n}{label}\PY{o}{=}\PY{l+s+sa}{r}\PY{l+s+s1}{\PYZsq{}}\PY{l+s+s1}{input \PYZdl{}I\PYZus{}2(t)\PYZdl{}}\PY{l+s+s1}{\PYZsq{}}\PY{p}{)}
\PY{n}{plt}\PY{o}{.}\PY{n}{plot}\PY{p}{(}\PY{n}{t}\PY{p}{,} \PY{n}{f3}\PY{p}{(}\PY{n}{t}\PY{p}{)}\PY{p}{,} \PY{n}{label}\PY{o}{=}\PY{l+s+sa}{r}\PY{l+s+s1}{\PYZsq{}}\PY{l+s+s1}{input \PYZdl{}I\PYZus{}3(t)\PYZdl{}}\PY{l+s+s1}{\PYZsq{}}\PY{p}{)}
\PY{n}{plt}\PY{o}{.}\PY{n}{xlabel}\PY{p}{(}\PY{l+s+s1}{\PYZsq{}}\PY{l+s+s1}{time}\PY{l+s+s1}{\PYZsq{}}\PY{p}{,} \PY{n}{fontsize}\PY{o}{=}\PY{l+m+mi}{14}\PY{p}{)}
\PY{n}{plt}\PY{o}{.}\PY{n}{ylabel}\PY{p}{(}\PY{l+s+s1}{\PYZsq{}}\PY{l+s+s1}{voltage}\PY{l+s+s1}{\PYZsq{}}\PY{p}{,} \PY{n}{fontsize}\PY{o}{=}\PY{l+m+mi}{14}\PY{p}{)}
\PY{n}{plt}\PY{o}{.}\PY{n}{legend}\PY{p}{(}\PY{p}{)}
\PY{n}{plt}\PY{o}{.}\PY{n}{show}\PY{p}{(}\PY{p}{)}
\end{Verbatim}
\end{tcolorbox}

    \begin{center}
    \adjustimage{max size={0.9\linewidth}{0.9\paperheight}}{output_7_0.png}
    \end{center}
    { \hspace*{\fill} \\}
    
    \subsection*{Additional note}\label{additional-note}

If we take again the differential equation, we define $u - E_L = v$
(hence $\frac{dv}{dt} = \frac{du}{dt}$) and we set $v(0) = 0$, the
solution of the differential equation in the Laplace domain becomes
\[ \tilde v(s) = \frac{1}{C_m \tau} \frac{\tilde I(s)}{s + 1/\tau} = \frac{1/\tau}{s+1/\tau} \frac{\tilde I(s)}{C_m} \equiv \frac{\omega_0}{s+\omega_0} \, \tilde v_{in}(s)\]
we get the transfer function
$H(s) = \frac{\tilde v_{out}(s)}{\tilde v_{in}(s)}$ of a low pass filter
with cutoff frequency $\omega_0 = \frac{1}{\tau}$. The transfer funciton
has just one pole and lies in the left half plane, hence the system is
stable, and indeed we saw in point a) that after a small perturbation,
it returns to the equilibrium point.

    \section*{Exercise 2}\label{exercise-2}

\subsubsection*{a)}\label{a}

\begin{equation*}
    C_m \dot u = g_L (E_L - u) + g_{exc} (E_{exc} - u) + g_{inh} (E_{inh} - u) + I_{ext}
\end{equation*}

    \begin{equation*}
    C_m \dot u = g_L E_L + g_{exc} E_{exc} + g_{inh} E_{inh} + I_{ext} - \left(g_L + g_{exc} + g_{inh}\right) u
\end{equation*}

which means

\begin{equation*}
    \dot u = \frac{g_L + g_{exc} + g_{inh}}{C_m} \ \left(\frac{g_L E_L + g_{exc} E_{exc} + g_{inh} E_{inh} + I_{ext}}{g_L + g_{exc} + g_{inh}} - u\right) \equiv \frac{1}{\tau_{eff}} \left(U_{eff} - u\right)
\end{equation*}

where we have set

\begin{equation*}
    \tau_{eff} \equiv \frac{C_m}{g_L + g_{exc} + g_{inh}} \qquad U_{eff} \equiv \frac{g_L E_L + g_{exc} E_{exc} + g_{inh} E_{inh} + I_{ext}}{g_L + g_{exc} + g_{inh}}
\end{equation*}

    \subsubsection*{b)}\label{b}

    Each term of the type $\frac{1}{\tau} \left(E - u\right)$ that appears
on the right hand side of the differential equation, has the effect of
"driving" u towards $E$ with a typical time $\tau$. When we sum up three
terms of this type, as in the differential equation of the COBA LIF, we
have that each of them drives $u$ towards a different voltage value
(respectively $E_L, E_{exc}, E_{inh}$) with different time constants
(respectively $\tau, \tau_{exc}, \tau_{inh}$). The net effect is
equivalent to just one term that drives $u$ towards
$U_{eff} = \frac{g_L E_L + g_{exc} E_{exc} + g_{inh} E_{inh} + I_{ext}}{g_L + g_{exc} + g_{inh}}$
with time constant
$\tau_{eff} = \frac{C_m}{g_L + g_{exc} + g_{inh}}$.\\Thus, this is not
always the case, since the excitatory and inhibitory conductances depend
on time and are proportional to functions of the type
\[ K(t) = \sum_{s} \theta(t - \Delta t_s) e^{(t - \Delta t_s)/\tau_i}\]
This means that when we get and inhibitory or excitatory spikes, the
corresponding conductances get "activated", but they exponentially decay
to a zero value with a time constant $\tau_{exc}, \tau_{inh}$.\\When
excitatory inputs are given, $U_{eff} > E_L$, hence the system is driven
towards a higher potential and results, eventually, in crossing the
threshold (we actually need two consectutive spikes for this). Thus, if
we give an inhibitory input before an excitatory one, we "active" the
inihibitory conductance, hence the term that drives the potential to a
lower value that $E_L$, and the net effect is that the system cannot
reach the treshold value. This is both due to that $E_{inh}$ is low
compared to $E_{th}$ but also because $\tau_{exc}$ is not long enough
compared to $\tau_{inh}$, otherwise the exciting term would have
"survived" also after the inhibitory one disappeared.

    \section*{Exercise 3}\label{exercise-3}

\subsection*{a)}\label{a}

We start by loading the dataset and plotting them

    \begin{tcolorbox}[breakable, size=fbox, boxrule=1pt, pad at break*=1mm,colback=cellbackground, colframe=cellborder]
\prompt{In}{incolor}{2}{\boxspacing}
\begin{Verbatim}[commandchars=\\\{\}]
\PY{k+kn}{import} \PY{n+nn}{numpy} \PY{k}{as} \PY{n+nn}{np}
\PY{k+kn}{import} \PY{n+nn}{matplotlib}\PY{n+nn}{.}\PY{n+nn}{pyplot} \PY{k}{as} \PY{n+nn}{plt}
\end{Verbatim}
\end{tcolorbox}

    \begin{tcolorbox}[breakable, size=fbox, boxrule=1pt, pad at break*=1mm,colback=cellbackground, colframe=cellborder]
\prompt{In}{incolor}{3}{\boxspacing}
\begin{Verbatim}[commandchars=\\\{\}]
\PY{c+c1}{\PYZsh{}loading the dataset}
\PY{n}{weak} \PY{o}{=} \PY{n}{np}\PY{o}{.}\PY{n}{load}\PY{p}{(}\PY{l+s+s1}{\PYZsq{}}\PY{l+s+s1}{cobaSynapse\PYZus{}V\PYZus{}weak.npy}\PY{l+s+s1}{\PYZsq{}}\PY{p}{)}
\PY{n}{medium} \PY{o}{=} \PY{n}{np}\PY{o}{.}\PY{n}{load}\PY{p}{(}\PY{l+s+s1}{\PYZsq{}}\PY{l+s+s1}{cobaSynapse\PYZus{}V\PYZus{}medium.npy}\PY{l+s+s1}{\PYZsq{}}\PY{p}{)}
\PY{n}{strong} \PY{o}{=} \PY{n}{np}\PY{o}{.}\PY{n}{load}\PY{p}{(}\PY{l+s+s1}{\PYZsq{}}\PY{l+s+s1}{cobaSynapse\PYZus{}V\PYZus{}strong.npy}\PY{l+s+s1}{\PYZsq{}}\PY{p}{)}


\PY{c+c1}{\PYZsh{}creates a time arry for the sampling}
\PY{n}{nu} \PY{o}{=} \PY{l+m+mi}{10}\PY{o}{*}\PY{o}{*}\PY{l+m+mi}{4}  \PY{c+c1}{\PYZsh{}Hz}
\PY{n}{tmax} \PY{o}{=} \PY{n+nb}{len}\PY{p}{(}\PY{n}{strong}\PY{p}{)}\PY{o}{/}\PY{n}{nu}
\PY{n}{t} \PY{o}{=} \PY{n}{np}\PY{o}{.}\PY{n}{arange}\PY{p}{(}\PY{l+m+mi}{0}\PY{p}{,}\PY{n}{tmax}\PY{p}{,} \PY{l+m+mi}{1}\PY{o}{/}\PY{n}{nu}\PY{p}{)}

\PY{c+c1}{\PYZsh{}creates time array for the probe spikes}
\PY{n}{ts} \PY{o}{=} \PY{n}{np}\PY{o}{.}\PY{n}{arange}\PY{p}{(}\PY{l+m+mf}{0.02}\PY{p}{,}\PY{n}{tmax}\PY{p}{,} \PY{l+m+mf}{0.02}\PY{p}{)}
\end{Verbatim}
\end{tcolorbox}

    \begin{tcolorbox}[breakable, size=fbox, boxrule=1pt, pad at break*=1mm,colback=cellbackground, colframe=cellborder]
\prompt{In}{incolor}{4}{\boxspacing}
\begin{Verbatim}[commandchars=\\\{\}]
\PY{c+c1}{\PYZsh{}plot of the noisy membrane potentials. Note that the y\PYZhy{}axis has the same scale and limits for all the plots}
\PY{n}{fig}\PY{p}{,} \PY{p}{(}\PY{n}{ax1}\PY{p}{,} \PY{n}{ax2}\PY{p}{,} \PY{n}{ax3}\PY{p}{)} \PY{o}{=} \PY{n}{plt}\PY{o}{.}\PY{n}{subplots}\PY{p}{(}\PY{l+m+mi}{3}\PY{p}{,} \PY{n}{sharex}\PY{o}{=}\PY{k+kc}{True}\PY{p}{)}


\PY{n}{ax1}\PY{o}{.}\PY{n}{set}\PY{p}{(}\PY{n}{ylabel}\PY{o}{=}\PY{l+s+s1}{\PYZsq{}}\PY{l+s+s1}{\PYZdl{}u\PYZus{}}\PY{l+s+si}{\PYZob{}w\PYZcb{}}\PY{l+s+s1}{(t)}\PY{l+s+s1}{\PYZbs{}}\PY{l+s+s1}{quad [mV]\PYZdl{}}\PY{l+s+s1}{\PYZsq{}}\PY{p}{,}\PY{n}{xlim}\PY{o}{=}\PY{p}{[}\PY{l+m+mi}{0}\PY{p}{,}\PY{l+m+mi}{2}\PY{p}{]}\PY{p}{,}\PY{n}{ylim}\PY{o}{=}\PY{p}{[}\PY{o}{\PYZhy{}}\PY{l+m+mi}{73}\PY{p}{,}\PY{o}{\PYZhy{}}\PY{l+m+mi}{64}\PY{p}{]}\PY{p}{)}   
\PY{n}{ax1}\PY{o}{.}\PY{n}{plot}\PY{p}{(}\PY{n}{t}\PY{p}{,}\PY{n}{weak}\PY{p}{,} \PY{n}{label} \PY{o}{=} \PY{l+s+s1}{\PYZsq{}}\PY{l+s+s1}{weak}\PY{l+s+s1}{\PYZsq{}}\PY{p}{)} \PY{c+c1}{\PYZsh{} plot of the analytical solution}
\PY{n}{ax1R} \PY{o}{=} \PY{n}{ax1}\PY{o}{.}\PY{n}{twinx}\PY{p}{(}\PY{p}{)}
\PY{n}{ax1T} \PY{o}{=} \PY{n}{ax1}\PY{o}{.}\PY{n}{twiny}\PY{p}{(}\PY{p}{)}
\PY{n}{ax1T}\PY{o}{.}\PY{n}{set}\PY{p}{(}\PY{n}{xlim}\PY{o}{=}\PY{p}{[}\PY{l+m+mi}{0}\PY{p}{,}\PY{l+m+mi}{2}\PY{p}{]}\PY{p}{)}
\PY{n}{ax1}\PY{o}{.}\PY{n}{tick\PYZus{}params}\PY{p}{(}\PY{n}{direction}\PY{o}{=}\PY{l+s+s1}{\PYZsq{}}\PY{l+s+s1}{in}\PY{l+s+s1}{\PYZsq{}}\PY{p}{)}
\PY{n}{ax1T}\PY{o}{.}\PY{n}{tick\PYZus{}params}\PY{p}{(}\PY{n}{direction}\PY{o}{=}\PY{l+s+s1}{\PYZsq{}}\PY{l+s+s1}{in}\PY{l+s+s1}{\PYZsq{}}\PY{p}{)}
\PY{n}{ax1R}\PY{o}{.}\PY{n}{tick\PYZus{}params}\PY{p}{(}\PY{n}{direction}\PY{o}{=}\PY{l+s+s1}{\PYZsq{}}\PY{l+s+s1}{in}\PY{l+s+s1}{\PYZsq{}}\PY{p}{)}
\PY{n}{ax1R}\PY{o}{.}\PY{n}{yaxis}\PY{o}{.}\PY{n}{set\PYZus{}major\PYZus{}formatter}\PY{p}{(}\PY{n}{plt}\PY{o}{.}\PY{n}{NullFormatter}\PY{p}{(}\PY{p}{)}\PY{p}{)}
\PY{n}{ax1T}\PY{o}{.}\PY{n}{xaxis}\PY{o}{.}\PY{n}{set\PYZus{}major\PYZus{}formatter}\PY{p}{(}\PY{n}{plt}\PY{o}{.}\PY{n}{NullFormatter}\PY{p}{(}\PY{p}{)}\PY{p}{)}

\PY{n}{ax2}\PY{o}{.}\PY{n}{set}\PY{p}{(}\PY{n}{ylabel}\PY{o}{=}\PY{l+s+s1}{\PYZsq{}}\PY{l+s+s1}{\PYZdl{}u\PYZus{}}\PY{l+s+si}{\PYZob{}m\PYZcb{}}\PY{l+s+s1}{(t)}\PY{l+s+s1}{\PYZbs{}}\PY{l+s+s1}{quad [mV]\PYZdl{}}\PY{l+s+s1}{\PYZsq{}}\PY{p}{,}\PY{n}{xlim}\PY{o}{=}\PY{p}{[}\PY{l+m+mi}{0}\PY{p}{,}\PY{l+m+mi}{2}\PY{p}{]}\PY{p}{,}\PY{n}{ylim}\PY{o}{=}\PY{p}{[}\PY{o}{\PYZhy{}}\PY{l+m+mi}{73}\PY{p}{,}\PY{o}{\PYZhy{}}\PY{l+m+mi}{64}\PY{p}{]}\PY{p}{)}   
\PY{n}{ax2}\PY{o}{.}\PY{n}{plot}\PY{p}{(}\PY{n}{t}\PY{p}{,}\PY{n}{medium}\PY{p}{,} \PY{n}{label} \PY{o}{=} \PY{l+s+s1}{\PYZsq{}}\PY{l+s+s1}{weak}\PY{l+s+s1}{\PYZsq{}}\PY{p}{)} \PY{c+c1}{\PYZsh{} plot of the analytical solution}
\PY{n}{ax2R} \PY{o}{=} \PY{n}{ax2}\PY{o}{.}\PY{n}{twinx}\PY{p}{(}\PY{p}{)}
\PY{n}{ax2T} \PY{o}{=} \PY{n}{ax2}\PY{o}{.}\PY{n}{twiny}\PY{p}{(}\PY{p}{)}
\PY{n}{ax2T}\PY{o}{.}\PY{n}{set}\PY{p}{(}\PY{n}{xlim}\PY{o}{=}\PY{p}{[}\PY{l+m+mi}{0}\PY{p}{,}\PY{l+m+mi}{2}\PY{p}{]}\PY{p}{)}
\PY{n}{ax2}\PY{o}{.}\PY{n}{tick\PYZus{}params}\PY{p}{(}\PY{n}{direction}\PY{o}{=}\PY{l+s+s1}{\PYZsq{}}\PY{l+s+s1}{in}\PY{l+s+s1}{\PYZsq{}}\PY{p}{)}
\PY{n}{ax2T}\PY{o}{.}\PY{n}{tick\PYZus{}params}\PY{p}{(}\PY{n}{direction}\PY{o}{=}\PY{l+s+s1}{\PYZsq{}}\PY{l+s+s1}{in}\PY{l+s+s1}{\PYZsq{}}\PY{p}{)}
\PY{n}{ax2R}\PY{o}{.}\PY{n}{tick\PYZus{}params}\PY{p}{(}\PY{n}{direction}\PY{o}{=}\PY{l+s+s1}{\PYZsq{}}\PY{l+s+s1}{in}\PY{l+s+s1}{\PYZsq{}}\PY{p}{)}
\PY{n}{ax2R}\PY{o}{.}\PY{n}{yaxis}\PY{o}{.}\PY{n}{set\PYZus{}major\PYZus{}formatter}\PY{p}{(}\PY{n}{plt}\PY{o}{.}\PY{n}{NullFormatter}\PY{p}{(}\PY{p}{)}\PY{p}{)}
\PY{n}{ax2T}\PY{o}{.}\PY{n}{xaxis}\PY{o}{.}\PY{n}{set\PYZus{}major\PYZus{}formatter}\PY{p}{(}\PY{n}{plt}\PY{o}{.}\PY{n}{NullFormatter}\PY{p}{(}\PY{p}{)}\PY{p}{)}

\PY{n}{ax3}\PY{o}{.}\PY{n}{set}\PY{p}{(}\PY{n}{xlabel}\PY{o}{=}\PY{l+s+s1}{\PYZsq{}}\PY{l+s+s1}{\PYZdl{}t }\PY{l+s+s1}{\PYZbs{}}\PY{l+s+s1}{quad [s]\PYZdl{}}\PY{l+s+s1}{\PYZsq{}}\PY{p}{,}\PY{n}{xlim}\PY{o}{=}\PY{p}{[}\PY{l+m+mi}{0}\PY{p}{,}\PY{l+m+mi}{2}\PY{p}{]}\PY{p}{,}\PY{n}{ylim}\PY{o}{=}\PY{p}{[}\PY{o}{\PYZhy{}}\PY{l+m+mi}{73}\PY{p}{,}\PY{o}{\PYZhy{}}\PY{l+m+mi}{64}\PY{p}{]}\PY{p}{)}
\PY{n}{ax3}\PY{o}{.}\PY{n}{set}\PY{p}{(}\PY{n}{ylabel}\PY{o}{=}\PY{l+s+s1}{\PYZsq{}}\PY{l+s+s1}{\PYZdl{}u\PYZus{}}\PY{l+s+si}{\PYZob{}s\PYZcb{}}\PY{l+s+s1}{(t)}\PY{l+s+s1}{\PYZbs{}}\PY{l+s+s1}{quad [mV]\PYZdl{}}\PY{l+s+s1}{\PYZsq{}}\PY{p}{)}   
\PY{n}{ax3}\PY{o}{.}\PY{n}{plot}\PY{p}{(}\PY{n}{t}\PY{p}{,}\PY{n}{strong}\PY{p}{,} \PY{n}{label} \PY{o}{=} \PY{l+s+s1}{\PYZsq{}}\PY{l+s+s1}{weak}\PY{l+s+s1}{\PYZsq{}}\PY{p}{)} \PY{c+c1}{\PYZsh{} plot of the analytical solution}
\PY{n}{ax3R} \PY{o}{=} \PY{n}{ax3}\PY{o}{.}\PY{n}{twinx}\PY{p}{(}\PY{p}{)}
\PY{n}{ax3T} \PY{o}{=} \PY{n}{ax3}\PY{o}{.}\PY{n}{twiny}\PY{p}{(}\PY{p}{)}
\PY{n}{ax3T}\PY{o}{.}\PY{n}{set}\PY{p}{(}\PY{n}{xlim}\PY{o}{=}\PY{p}{[}\PY{l+m+mi}{0}\PY{p}{,}\PY{l+m+mi}{2}\PY{p}{]}\PY{p}{)}
\PY{n}{ax3}\PY{o}{.}\PY{n}{tick\PYZus{}params}\PY{p}{(}\PY{n}{direction}\PY{o}{=}\PY{l+s+s1}{\PYZsq{}}\PY{l+s+s1}{in}\PY{l+s+s1}{\PYZsq{}}\PY{p}{)}
\PY{n}{ax3T}\PY{o}{.}\PY{n}{tick\PYZus{}params}\PY{p}{(}\PY{n}{direction}\PY{o}{=}\PY{l+s+s1}{\PYZsq{}}\PY{l+s+s1}{in}\PY{l+s+s1}{\PYZsq{}}\PY{p}{)}
\PY{n}{ax3R}\PY{o}{.}\PY{n}{tick\PYZus{}params}\PY{p}{(}\PY{n}{direction}\PY{o}{=}\PY{l+s+s1}{\PYZsq{}}\PY{l+s+s1}{in}\PY{l+s+s1}{\PYZsq{}}\PY{p}{)}
\PY{n}{ax3R}\PY{o}{.}\PY{n}{yaxis}\PY{o}{.}\PY{n}{set\PYZus{}major\PYZus{}formatter}\PY{p}{(}\PY{n}{plt}\PY{o}{.}\PY{n}{NullFormatter}\PY{p}{(}\PY{p}{)}\PY{p}{)}
\PY{n}{ax3T}\PY{o}{.}\PY{n}{xaxis}\PY{o}{.}\PY{n}{set\PYZus{}major\PYZus{}formatter}\PY{p}{(}\PY{n}{plt}\PY{o}{.}\PY{n}{NullFormatter}\PY{p}{(}\PY{p}{)}\PY{p}{)}

\PY{n}{plt}\PY{o}{.}\PY{n}{subplots\PYZus{}adjust}\PY{p}{(}\PY{n}{wspace}\PY{o}{=}\PY{l+m+mi}{0}\PY{p}{,} \PY{n}{hspace}\PY{o}{=}\PY{l+m+mi}{0}\PY{p}{)}
\PY{n}{plt}\PY{o}{.}\PY{n}{show}\PY{p}{(}\PY{p}{)}
\end{Verbatim}
\end{tcolorbox}

    \begin{center}
    \adjustimage{max size={0.9\linewidth}{0.9\paperheight}}{output_16_0.png}
    \end{center}
    { \hspace*{\fill} \\}
    
    \subsection*{b)}\label{b}

We now average by time aligment on the spike train

    \begin{tcolorbox}[breakable, size=fbox, boxrule=1pt, pad at break*=1mm,colback=cellbackground, colframe=cellborder]
\prompt{In}{incolor}{5}{\boxspacing}
\begin{Verbatim}[commandchars=\\\{\}]
\PY{n}{t} \PY{o}{=} \PY{n}{np}\PY{o}{.}\PY{n}{arange}\PY{p}{(}\PY{l+m+mi}{0}\PY{p}{,}\PY{l+m+mf}{0.02}\PY{p}{,} \PY{l+m+mi}{1}\PY{o}{/}\PY{n}{nu}\PY{p}{)}
\PY{n}{N} \PY{o}{=} \PY{n+nb}{int}\PY{p}{(}\PY{n+nb}{len}\PY{p}{(}\PY{n}{weak}\PY{p}{)}\PY{o}{/}\PY{n+nb}{len}\PY{p}{(}\PY{n}{t}\PY{p}{)}\PY{p}{)}

\PY{n}{mean\PYZus{}weak} \PY{o}{=} \PY{n}{np}\PY{o}{.}\PY{n}{zeros}\PY{p}{(}\PY{n+nb}{len}\PY{p}{(}\PY{n}{t}\PY{p}{)}\PY{p}{)}
\PY{n}{mean\PYZus{}medium} \PY{o}{=} \PY{n}{np}\PY{o}{.}\PY{n}{zeros}\PY{p}{(}\PY{n+nb}{len}\PY{p}{(}\PY{n}{t}\PY{p}{)}\PY{p}{)}
\PY{n}{mean\PYZus{}strong} \PY{o}{=} \PY{n}{np}\PY{o}{.}\PY{n}{zeros}\PY{p}{(}\PY{n+nb}{len}\PY{p}{(}\PY{n}{t}\PY{p}{)}\PY{p}{)}

\PY{k}{for} \PY{n}{n} \PY{o+ow}{in} \PY{n+nb}{range}\PY{p}{(}\PY{n}{N}\PY{p}{)}\PY{p}{:}
    \PY{n}{mean\PYZus{}weak} \PY{o}{+}\PY{o}{=} \PY{n}{weak}\PY{p}{[}\PY{n}{n}\PY{p}{:}\PY{n}{n}\PY{o}{+}\PY{n+nb}{len}\PY{p}{(}\PY{n}{t}\PY{p}{)}\PY{p}{]}
    \PY{n}{mean\PYZus{}medium} \PY{o}{+}\PY{o}{=} \PY{n}{medium}\PY{p}{[}\PY{n}{n}\PY{p}{:}\PY{n}{n}\PY{o}{+}\PY{n+nb}{len}\PY{p}{(}\PY{n}{t}\PY{p}{)}\PY{p}{]}
    \PY{n}{mean\PYZus{}strong} \PY{o}{+}\PY{o}{=} \PY{n}{strong}\PY{p}{[}\PY{n}{n}\PY{p}{:}\PY{n}{n}\PY{o}{+}\PY{n+nb}{len}\PY{p}{(}\PY{n}{t}\PY{p}{)}\PY{p}{]}
\end{Verbatim}
\end{tcolorbox}

    \subsection*{c)}\label{c}

we now plot the averaged dataset.

    \begin{tcolorbox}[breakable, size=fbox, boxrule=1pt, pad at break*=1mm,colback=cellbackground, colframe=cellborder]
\prompt{In}{incolor}{6}{\boxspacing}
\begin{Verbatim}[commandchars=\\\{\}]
\PY{n}{plt}\PY{o}{.}\PY{n}{plot}\PY{p}{(}\PY{n}{t}\PY{p}{,} \PY{n}{mean\PYZus{}weak}\PY{p}{,} \PY{n}{label}\PY{o}{=} \PY{l+s+s1}{\PYZsq{}}\PY{l+s+s1}{weak}\PY{l+s+s1}{\PYZsq{}}\PY{p}{,} \PY{n}{linewidth}\PY{o}{=} \PY{l+m+mi}{1}\PY{p}{)}
\PY{n}{plt}\PY{o}{.}\PY{n}{plot}\PY{p}{(}\PY{n}{t}\PY{p}{,} \PY{n}{mean\PYZus{}medium}\PY{p}{,}\PY{n}{label}\PY{o}{=} \PY{l+s+s1}{\PYZsq{}}\PY{l+s+s1}{medium}\PY{l+s+s1}{\PYZsq{}}\PY{p}{,}\PY{n}{linewidth}\PY{o}{=} \PY{l+m+mi}{1}\PY{p}{)}
\PY{n}{plt}\PY{o}{.}\PY{n}{plot}\PY{p}{(}\PY{n}{t}\PY{p}{,} \PY{n}{mean\PYZus{}strong}\PY{p}{,}\PY{n}{label}\PY{o}{=} \PY{l+s+s1}{\PYZsq{}}\PY{l+s+s1}{strong}\PY{l+s+s1}{\PYZsq{}}\PY{p}{,}\PY{n}{linewidth}\PY{o}{=} \PY{l+m+mi}{1}\PY{p}{)}
\PY{n}{plt}\PY{o}{.}\PY{n}{legend}\PY{p}{(}\PY{p}{)}
\PY{n}{plt}\PY{o}{.}\PY{n}{grid}\PY{p}{(}\PY{p}{)}
\PY{n}{plt}\PY{o}{.}\PY{n}{show}\PY{p}{(}\PY{p}{)}
    
    
\end{Verbatim}
\end{tcolorbox}

    \begin{center}
    \adjustimage{max size={0.9\linewidth}{0.9\paperheight}}{output_20_0.png}
    \end{center}
    { \hspace*{\fill} \\}
    
    \subsection*{d)}\label{d}

We can note that the strong background is the most rapid in responding
to the spike, and that the low one is the slowest. That means that the
times constants decrease with the background activity. We can also note
taht the aplitude of the strong is bigger, and that is the only one that
starts to decrease after the response. The amplitude of the
low-backgound neuron response is significantly smaller than the others
two, that are comparable.

    \section*{Exercise 4}\label{exercise-4}

    \subsection*{a)}\label{a}

We start by simply running the given nest template without canging the
parameters:

    \begin{tcolorbox}[breakable, size=fbox, boxrule=1pt, pad at break*=1mm,colback=cellbackground, colframe=cellborder]
\prompt{In}{incolor}{7}{\boxspacing}
\begin{Verbatim}[commandchars=\\\{\}]
\PY{k+kn}{import} \PY{n+nn}{nest}
\PY{k+kn}{import} \PY{n+nn}{numpy} \PY{k}{as} \PY{n+nn}{np}
\PY{k+kn}{import} \PY{n+nn}{matplotlib}\PY{n+nn}{.}\PY{n+nn}{pyplot} \PY{k}{as} \PY{n+nn}{plt}
\PY{k+kn}{from} \PY{n+nn}{scipy} \PY{k+kn}{import} \PY{n}{optimize}
\end{Verbatim}
\end{tcolorbox}

    \begin{Verbatim}[commandchars=\\\{\}]

              -- N E S T --
  Copyright (C) 2004 The NEST Initiative

 Version: nest-3.0
 Built: Oct  9 2021 08:53:03

 This program is provided AS IS and comes with
 NO WARRANTY. See the file LICENSE for details.

 Problems or suggestions?
   Visit https://www.nest-simulator.org

 Type 'nest.help()' to find out more about NEST.


Nov 16 10:57:26 evalrcfile [Warning]:
    NEST has created its configuration file \$HOME/.nestrc

Nov 16 10:57:26 evalrcfile [Warning]:
    If this troubles you, type '/nestrc help' at the prompt to learn more.
    \end{Verbatim}

    \begin{tcolorbox}[breakable, size=fbox, boxrule=1pt, pad at break*=1mm,colback=cellbackground, colframe=cellborder]
\prompt{In}{incolor}{8}{\boxspacing}
\begin{Verbatim}[commandchars=\\\{\}]
\PY{k}{def} \PY{n+nf}{run\PYZus{}network}\PY{p}{(}\PY{n}{vrest}\PY{p}{,} \PY{n}{cm}\PY{p}{,} \PY{n}{ereve}\PY{p}{,} \PY{n}{erevi}\PY{p}{,} \PY{n}{gl}\PY{p}{,} \PY{n}{tausyne}\PY{p}{,} \PY{n}{tausyni}\PY{p}{,} \PY{n}{wconde}\PY{p}{,} \PY{n}{wcondi}\PY{p}{)}\PY{p}{:}
    \PY{n}{nest}\PY{o}{.}\PY{n}{set\PYZus{}verbosity}\PY{p}{(}\PY{l+s+s2}{\PYZdq{}}\PY{l+s+s2}{M\PYZus{}ERROR}\PY{l+s+s2}{\PYZdq{}}\PY{p}{)}
    \PY{n}{spkte} \PY{o}{=} \PY{n}{np}\PY{o}{.}\PY{n}{array}\PY{p}{(}\PY{p}{[}\PY{l+m+mf}{50.}\PY{p}{,} \PY{l+m+mf}{100.}\PY{p}{,} \PY{l+m+mf}{103.}\PY{p}{,} \PY{l+m+mf}{106.}\PY{p}{,} \PY{l+m+mf}{109.}\PY{p}{,} \PY{l+m+mf}{112.}\PY{p}{,} \PY{l+m+mf}{115.}\PY{p}{,} \PY{l+m+mf}{118.}\PY{p}{,} \PY{l+m+mf}{121.}\PY{p}{,} \PY{l+m+mf}{124.}\PY{p}{,} \PY{l+m+mf}{127.}\PY{p}{]}\PY{p}{)} \PY{o}{+} \PY{l+m+mf}{150.}
    \PY{n}{spkti} \PY{o}{=} \PY{n}{spkte} \PY{o}{+} \PY{l+m+mf}{150.}
    
    \PY{n}{neuron} \PY{o}{=} \PY{n}{nest}\PY{o}{.}\PY{n}{Create}\PY{p}{(}\PY{l+s+s1}{\PYZsq{}}\PY{l+s+s1}{iaf\PYZus{}cond\PYZus{}exp}\PY{l+s+s1}{\PYZsq{}}\PY{p}{,} \PY{p}{\PYZob{}}\PY{l+s+s1}{\PYZsq{}}\PY{l+s+s1}{V\PYZus{}m}\PY{l+s+s1}{\PYZsq{}}\PY{p}{:} \PY{n}{vrest}\PY{p}{,} \PY{l+s+s1}{\PYZsq{}}\PY{l+s+s1}{E\PYZus{}L}\PY{l+s+s1}{\PYZsq{}}\PY{p}{:} \PY{n}{vrest}\PY{p}{,} \PY{l+s+s1}{\PYZsq{}}\PY{l+s+s1}{C\PYZus{}m}\PY{l+s+s1}{\PYZsq{}}\PY{p}{:} \PY{n}{cm}\PY{p}{,} \PY{l+s+s1}{\PYZsq{}}\PY{l+s+s1}{E\PYZus{}ex}\PY{l+s+s1}{\PYZsq{}}\PY{p}{:} \PY{n}{ereve}\PY{p}{,} \PY{l+s+s1}{\PYZsq{}}\PY{l+s+s1}{E\PYZus{}in}\PY{l+s+s1}{\PYZsq{}}\PY{p}{:} \PY{n}{erevi}\PY{p}{,} \PY{l+s+s1}{\PYZsq{}}\PY{l+s+s1}{g\PYZus{}L}\PY{l+s+s1}{\PYZsq{}}\PY{p}{:} \PY{n}{gl}\PY{p}{,} \PY{l+s+s1}{\PYZsq{}}\PY{l+s+s1}{tau\PYZus{}syn\PYZus{}ex}\PY{l+s+s1}{\PYZsq{}}\PY{p}{:} \PY{n}{tausyne}\PY{p}{,} \PY{l+s+s1}{\PYZsq{}}\PY{l+s+s1}{tau\PYZus{}syn\PYZus{}in}\PY{l+s+s1}{\PYZsq{}}\PY{p}{:} \PY{n}{tausyni}\PY{p}{,} \PY{l+s+s1}{\PYZsq{}}\PY{l+s+s1}{V\PYZus{}th}\PY{l+s+s1}{\PYZsq{}}\PY{p}{:}\PY{l+m+mf}{100.}\PY{p}{\PYZcb{}}\PY{p}{)}
    \PY{n}{spikegeneratore} \PY{o}{=} \PY{n}{nest}\PY{o}{.}\PY{n}{Create}\PY{p}{(}\PY{l+s+s1}{\PYZsq{}}\PY{l+s+s1}{spike\PYZus{}generator}\PY{l+s+s1}{\PYZsq{}}\PY{p}{)}
    \PY{n}{nest}\PY{o}{.}\PY{n}{SetStatus}\PY{p}{(}\PY{n}{spikegeneratore}\PY{p}{,} \PY{p}{\PYZob{}}\PY{l+s+s1}{\PYZsq{}}\PY{l+s+s1}{spike\PYZus{}times}\PY{l+s+s1}{\PYZsq{}}\PY{p}{:} \PY{n}{spkte}\PY{p}{\PYZcb{}}\PY{p}{)}
    \PY{n}{spikegeneratori} \PY{o}{=} \PY{n}{nest}\PY{o}{.}\PY{n}{Create}\PY{p}{(}\PY{l+s+s1}{\PYZsq{}}\PY{l+s+s1}{spike\PYZus{}generator}\PY{l+s+s1}{\PYZsq{}}\PY{p}{)}
    \PY{n}{nest}\PY{o}{.}\PY{n}{SetStatus}\PY{p}{(}\PY{n}{spikegeneratori}\PY{p}{,} \PY{p}{\PYZob{}}\PY{l+s+s1}{\PYZsq{}}\PY{l+s+s1}{spike\PYZus{}times}\PY{l+s+s1}{\PYZsq{}}\PY{p}{:} \PY{n}{spkti}\PY{p}{\PYZcb{}}\PY{p}{)}

    \PY{n}{voltmeter} \PY{o}{=} \PY{n}{nest}\PY{o}{.}\PY{n}{Create}\PY{p}{(}\PY{l+s+s1}{\PYZsq{}}\PY{l+s+s1}{voltmeter}\PY{l+s+s1}{\PYZsq{}}\PY{p}{,} \PY{p}{\PYZob{}}\PY{l+s+s1}{\PYZsq{}}\PY{l+s+s1}{interval}\PY{l+s+s1}{\PYZsq{}} \PY{p}{:}\PY{l+m+mf}{0.1}\PY{p}{\PYZcb{}}\PY{p}{)}
    \PY{n}{nest}\PY{o}{.}\PY{n}{Connect}\PY{p}{(}\PY{n}{spikegeneratore}\PY{p}{,} \PY{n}{neuron}\PY{p}{,} \PY{n}{syn\PYZus{}spec}\PY{o}{=}\PY{p}{\PYZob{}}\PY{l+s+s1}{\PYZsq{}}\PY{l+s+s1}{weight}\PY{l+s+s1}{\PYZsq{}}\PY{p}{:} \PY{n}{wconde}\PY{p}{\PYZcb{}}\PY{p}{)}
    \PY{n}{nest}\PY{o}{.}\PY{n}{Connect}\PY{p}{(}\PY{n}{spikegeneratori}\PY{p}{,} \PY{n}{neuron}\PY{p}{,} \PY{n}{syn\PYZus{}spec}\PY{o}{=}\PY{p}{\PYZob{}}\PY{l+s+s1}{\PYZsq{}}\PY{l+s+s1}{weight}\PY{l+s+s1}{\PYZsq{}}\PY{p}{:} \PY{o}{\PYZhy{}}\PY{n}{wcondi}\PY{p}{\PYZcb{}}\PY{p}{)}
    \PY{n}{nest}\PY{o}{.}\PY{n}{Connect}\PY{p}{(}\PY{n}{voltmeter}\PY{p}{,} \PY{n}{neuron}\PY{p}{)}
    \PY{n}{nest}\PY{o}{.}\PY{n}{print\PYZus{}time} \PY{o}{=} \PY{k+kc}{True}

    \PY{n}{nest}\PY{o}{.}\PY{n}{Simulate}\PY{p}{(}\PY{l+m+mf}{50.1}\PY{p}{)}
    \PY{n}{nest}\PY{o}{.}\PY{n}{SetStatus}\PY{p}{(}\PY{n}{neuron}\PY{p}{,} \PY{p}{\PYZob{}}\PY{l+s+s1}{\PYZsq{}}\PY{l+s+s1}{I\PYZus{}e}\PY{l+s+s1}{\PYZsq{}}\PY{p}{:} \PY{l+m+mf}{0.2}\PY{p}{\PYZcb{}}\PY{p}{)}
    \PY{n}{nest}\PY{o}{.}\PY{n}{Simulate}\PY{p}{(}\PY{l+m+mf}{100.1}\PY{p}{)}
    \PY{n}{nest}\PY{o}{.}\PY{n}{SetStatus}\PY{p}{(}\PY{n}{neuron}\PY{p}{,} \PY{p}{\PYZob{}}\PY{l+s+s1}{\PYZsq{}}\PY{l+s+s1}{I\PYZus{}e}\PY{l+s+s1}{\PYZsq{}}\PY{p}{:} \PY{l+m+mf}{0.}\PY{p}{\PYZcb{}}\PY{p}{)}
    \PY{n}{nest}\PY{o}{.}\PY{n}{Simulate}\PY{p}{(}\PY{l+m+mf}{350.1}\PY{p}{)}

    \PY{n}{voltage} \PY{o}{=} \PY{n}{nest}\PY{o}{.}\PY{n}{GetStatus}\PY{p}{(}\PY{n}{voltmeter}\PY{p}{,} \PY{l+s+s1}{\PYZsq{}}\PY{l+s+s1}{events}\PY{l+s+s1}{\PYZsq{}}\PY{p}{)}\PY{p}{[}\PY{l+m+mi}{0}\PY{p}{]}\PY{p}{[}\PY{l+s+s1}{\PYZsq{}}\PY{l+s+s1}{V\PYZus{}m}\PY{l+s+s1}{\PYZsq{}}\PY{p}{]}

    \PY{n}{nest}\PY{o}{.}\PY{n}{ResetKernel}\PY{p}{(}\PY{p}{)}
    \PY{k}{return} \PY{n}{voltage}
\end{Verbatim}
\end{tcolorbox}

    \begin{tcolorbox}[breakable, size=fbox, boxrule=1pt, pad at break*=1mm,colback=cellbackground, colframe=cellborder]
\prompt{In}{incolor}{9}{\boxspacing}
\begin{Verbatim}[commandchars=\\\{\}]
\PY{n}{data} \PY{o}{=} \PY{n}{np}\PY{o}{.}\PY{n}{load}\PY{p}{(}\PY{l+s+s1}{\PYZsq{}}\PY{l+s+s1}{membrane\PYZus{}trace\PYZus{}3.4\PYZus{}fixed.npy}\PY{l+s+s1}{\PYZsq{}}\PY{p}{)}\PY{p}{[}\PY{l+m+mi}{1}\PY{p}{:}\PY{p}{]}
\end{Verbatim}
\end{tcolorbox}

    \begin{tcolorbox}[breakable, size=fbox, boxrule=1pt, pad at break*=1mm,colback=cellbackground, colframe=cellborder]
\prompt{In}{incolor}{10}{\boxspacing}
\begin{Verbatim}[commandchars=\\\{\}]
\PY{k}{def} \PY{n+nf}{plot\PYZus{}comparison}\PY{p}{(}\PY{n}{v}\PY{p}{)}\PY{p}{:}
    \PY{k}{global} \PY{n}{data}
    \PY{n}{plt}\PY{o}{.}\PY{n}{figure}\PY{p}{(}\PY{p}{)}
    \PY{n}{t} \PY{o}{=} \PY{n}{np}\PY{o}{.}\PY{n}{arange}\PY{p}{(}\PY{l+m+mi}{0}\PY{p}{,}\PY{n+nb}{len}\PY{p}{(}\PY{n}{v}\PY{p}{)}\PY{o}{*}\PY{n}{dt}\PY{p}{,}\PY{n}{dt}\PY{p}{)}
    \PY{n}{plt}\PY{o}{.}\PY{n}{plot}\PY{p}{(}\PY{n}{t}\PY{p}{,} \PY{n}{v}\PY{p}{)}
    \PY{n}{t} \PY{o}{=} \PY{n}{np}\PY{o}{.}\PY{n}{arange}\PY{p}{(}\PY{l+m+mi}{0}\PY{p}{,} \PY{n+nb}{len}\PY{p}{(}\PY{n}{data}\PY{p}{)}\PY{o}{*}\PY{l+m+mf}{0.1}\PY{p}{,} \PY{l+m+mf}{0.1}\PY{p}{)}
    \PY{n}{plt}\PY{o}{.}\PY{n}{plot}\PY{p}{(}\PY{n}{t}\PY{p}{,} \PY{n}{data}\PY{p}{)}
\end{Verbatim}
\end{tcolorbox}

    \begin{tcolorbox}[breakable, size=fbox, boxrule=1pt, pad at break*=1mm,colback=cellbackground, colframe=cellborder]
\prompt{In}{incolor}{11}{\boxspacing}
\begin{Verbatim}[commandchars=\\\{\}]
\PY{c+c1}{\PYZsh{}known neuron parameters}
\PY{n}{cm} \PY{o}{=} \PY{l+m+mf}{1.} \PY{c+c1}{\PYZsh{} nF}
\PY{n}{ereve} \PY{o}{=} \PY{l+m+mf}{20.} \PY{c+c1}{\PYZsh{} mV}
\PY{n}{erevi} \PY{o}{=} \PY{o}{\PYZhy{}}\PY{l+m+mf}{80.} \PY{c+c1}{\PYZsh{} mV}

\PY{c+c1}{\PYZsh{} unknown neuron paramters (random values insertet here)}
\PY{n}{vrest} \PY{o}{=} \PY{o}{\PYZhy{}}\PY{l+m+mf}{50.} \PY{c+c1}{\PYZsh{} mV}
\PY{n}{wconde} \PY{o}{=} \PY{l+m+mf}{.0031416} \PY{c+c1}{\PYZsh{} uS}
\PY{n}{wcondi} \PY{o}{=} \PY{l+m+mf}{.0031416} \PY{c+c1}{\PYZsh{} uS}
\PY{n}{gl} \PY{o}{=} \PY{l+m+mf}{0.042} \PY{c+c1}{\PYZsh{} uS}
\PY{n}{tausyne} \PY{o}{=} \PY{l+m+mf}{3.1416} \PY{c+c1}{\PYZsh{} ms}
\PY{n}{tausyni} \PY{o}{=} \PY{l+m+mi}{2}\PY{o}{*}\PY{l+m+mf}{3.1416} \PY{c+c1}{\PYZsh{} ms}

\PY{n}{tsim} \PY{o}{=} \PY{l+m+mf}{500.}
\PY{n}{dt} \PY{o}{=} \PY{l+m+mf}{0.1}

\PY{n}{v} \PY{o}{=} \PY{n}{run\PYZus{}network}\PY{p}{(}\PY{n}{vrest}\PY{p}{,} \PY{n}{cm}\PY{p}{,} \PY{n}{ereve}\PY{p}{,} \PY{n}{erevi}\PY{p}{,} \PY{n}{gl}\PY{p}{,} \PY{n}{tausyne}\PY{p}{,} \PY{n}{tausyni}\PY{p}{,} \PY{n}{wconde}\PY{p}{,} \PY{n}{wcondi}\PY{p}{)}

\PY{n}{plot\PYZus{}comparison}\PY{p}{(}\PY{n}{v}\PY{p}{)}
\end{Verbatim}
\end{tcolorbox}

    \begin{center}
    \adjustimage{max size={0.9\linewidth}{0.9\paperheight}}{output_28_0.png}
    \end{center}
    { \hspace*{\fill} \\}
    
    The most evident difference is the shift in the resting potential of
about 20 mV. Then we can see that the amplitude of the displacemets
around the resting potential is too large, and we should then increase
$g_l$. We also note that the time constant for the exponential
excitatory rise is too large, as well as the one for the exponential
hinibitory decrease. We should then reduce $\tau_\text{eff}$ by
increeasing the exciting and hinibitory conductance weights.

    \subsection*{b)}\label{b}

We now try to change manually the parameters, following the observations
made above:

    \begin{tcolorbox}[breakable, size=fbox, boxrule=1pt, pad at break*=1mm,colback=cellbackground, colframe=cellborder]
\prompt{In}{incolor}{12}{\boxspacing}
\begin{Verbatim}[commandchars=\\\{\}]
\PY{c+c1}{\PYZsh{}known neuron parameters}
\PY{n}{cm} \PY{o}{=} \PY{l+m+mf}{1.} \PY{c+c1}{\PYZsh{} nF}
\PY{n}{ereve} \PY{o}{=} \PY{l+m+mf}{20.} \PY{c+c1}{\PYZsh{} mV}
\PY{n}{erevi} \PY{o}{=} \PY{o}{\PYZhy{}}\PY{l+m+mf}{80.} \PY{c+c1}{\PYZsh{} mV}

\PY{c+c1}{\PYZsh{} unknown neuron paramters (random values insertet here)}
\PY{n}{vrest} \PY{o}{=} \PY{o}{\PYZhy{}}\PY{l+m+mf}{70.} \PY{c+c1}{\PYZsh{} mV}
\PY{n}{wconde} \PY{o}{=} \PY{l+m+mf}{.0031416}\PY{o}{*}\PY{l+m+mf}{1.8} \PY{c+c1}{\PYZsh{} uS}
\PY{n}{wcondi} \PY{o}{=} \PY{l+m+mf}{.0031416}\PY{o}{*}\PY{l+m+mf}{9.5} \PY{c+c1}{\PYZsh{} uS}
\PY{n}{gl} \PY{o}{=} \PY{l+m+mf}{0.042}\PY{o}{*}\PY{l+m+mf}{2.5} \PY{c+c1}{\PYZsh{} uS}
\PY{n}{tausyne} \PY{o}{=} \PY{l+m+mf}{3.1416}\PY{o}{*}\PY{l+m+mf}{0.6} \PY{c+c1}{\PYZsh{} ms}
\PY{n}{tausyni} \PY{o}{=} \PY{l+m+mi}{2}\PY{o}{*}\PY{l+m+mf}{3.1416}\PY{o}{*}\PY{l+m+mf}{1.4}\PY{c+c1}{\PYZsh{} ms}

\PY{n}{tsim} \PY{o}{=} \PY{l+m+mf}{500.}
\PY{n}{dt} \PY{o}{=} \PY{l+m+mf}{0.1}

\PY{n}{v} \PY{o}{=} \PY{n}{run\PYZus{}network}\PY{p}{(}\PY{n}{vrest}\PY{p}{,} \PY{n}{cm}\PY{p}{,} \PY{n}{ereve}\PY{p}{,} \PY{n}{erevi}\PY{p}{,} \PY{n}{gl}\PY{p}{,} \PY{n}{tausyne}\PY{p}{,} \PY{n}{tausyni}\PY{p}{,} \PY{n}{wconde}\PY{p}{,} \PY{n}{wcondi}\PY{p}{)}

\PY{n}{plot\PYZus{}comparison}\PY{p}{(}\PY{n}{v}\PY{p}{)}
\end{Verbatim}
\end{tcolorbox}

    \begin{center}
    \adjustimage{max size={0.9\linewidth}{0.9\paperheight}}{output_31_0.png}
    \end{center}
    { \hspace*{\fill} \\}
    
    We can note that by doing what observed in point a), and modifing in
addition the rise and fall time constant, we obtain a far better
agreement with the desired potential.

    \subsection*{c)}\label{c}

We now optimize the parameters automatically:

    \begin{tcolorbox}[breakable, size=fbox, boxrule=1pt, pad at break*=1mm,colback=cellbackground, colframe=cellborder]
\prompt{In}{incolor}{13}{\boxspacing}
\begin{Verbatim}[commandchars=\\\{\}]
\PY{c+c1}{\PYZsh{} compute the total squared difference between data and the simulated potential. Take as input the parameters to optimize. }
\PY{k}{def} \PY{n+nf}{distance}\PY{p}{(}\PY{n}{params}\PY{p}{)}\PY{p}{:}
    \PY{n}{vrest} \PY{o}{=} \PY{n}{params}\PY{p}{[}\PY{l+m+mi}{0}\PY{p}{]}
    \PY{n}{gl} \PY{o}{=} \PY{n}{params}\PY{p}{[}\PY{l+m+mi}{1}\PY{p}{]}
    \PY{n}{tausyne} \PY{o}{=} \PY{n}{params}\PY{p}{[}\PY{l+m+mi}{2}\PY{p}{]} 
    \PY{n}{tausyni} \PY{o}{=} \PY{n}{params}\PY{p}{[}\PY{l+m+mi}{3}\PY{p}{]} 
    \PY{n}{wconde} \PY{o}{=} \PY{n}{params}\PY{p}{[}\PY{l+m+mi}{4}\PY{p}{]} 
    \PY{n}{wcondi} \PY{o}{=} \PY{n}{params}\PY{p}{[}\PY{l+m+mi}{5}\PY{p}{]}
    \PY{k}{return} \PY{n}{np}\PY{o}{.}\PY{n}{sum}\PY{p}{(}\PY{n}{np}\PY{o}{.}\PY{n}{power}\PY{p}{(}\PY{n}{data} \PY{o}{\PYZhy{}} \PY{n}{run\PYZus{}network}\PY{p}{(}\PY{n}{vrest}\PY{p}{,} \PY{n}{cm}\PY{p}{,} \PY{n}{ereve}\PY{p}{,} \PY{n}{erevi}\PY{p}{,} \PY{n}{gl}\PY{p}{,} \PY{n+nb}{abs}\PY{p}{(}\PY{n}{tausyne}\PY{p}{)}\PY{p}{,} \PY{n+nb}{abs}\PY{p}{(}\PY{n}{tausyni}\PY{p}{)}\PY{p}{,} \PY{n}{wconde}\PY{p}{,} \PY{n}{wcondi}\PY{p}{)}\PY{p}{,} \PY{l+m+mi}{2}\PY{p}{)}\PY{p}{)}
\end{Verbatim}
\end{tcolorbox}

    \begin{tcolorbox}[breakable, size=fbox, boxrule=1pt, pad at break*=1mm,colback=cellbackground, colframe=cellborder]
\prompt{In}{incolor}{14}{\boxspacing}
\begin{Verbatim}[commandchars=\\\{\}]
\PY{k+kn}{from} \PY{n+nn}{scipy}\PY{n+nn}{.}\PY{n+nn}{optimize} \PY{k+kn}{import} \PY{n}{minimize} \PY{k}{as} \PY{n}{fit}


\PY{n}{res} \PY{o}{=} \PY{n}{fit}\PY{p}{(}\PY{n}{distance}\PY{p}{,} \PY{n}{np}\PY{o}{.}\PY{n}{array}\PY{p}{(}\PY{p}{[}\PY{n}{vrest}\PY{p}{,} \PY{n}{gl}\PY{p}{,} \PY{n}{tausyne}\PY{p}{,} \PY{n}{tausyni}\PY{p}{,} \PY{n}{wconde}\PY{p}{,} \PY{n}{wcondi}\PY{p}{]}\PY{p}{)}\PY{p}{)}
\end{Verbatim}
\end{tcolorbox}

    \begin{tcolorbox}[breakable, size=fbox, boxrule=1pt, pad at break*=1mm,colback=cellbackground, colframe=cellborder]
\prompt{In}{incolor}{15}{\boxspacing}
\begin{Verbatim}[commandchars=\\\{\}]
\PY{n+nb}{print}\PY{p}{(}\PY{l+s+s2}{\PYZdq{}}\PY{l+s+s2}{The optimazed parameters are:}\PY{l+s+s2}{\PYZdq{}}\PY{p}{)}
\PY{n+nb}{print}\PY{p}{(}\PY{l+s+s2}{\PYZdq{}}\PY{l+s+s2}{Vrest =}\PY{l+s+s2}{\PYZdq{}}\PY{p}{,} \PY{n}{res}\PY{o}{.}\PY{n}{x}\PY{p}{[}\PY{l+m+mi}{0}\PY{p}{]}\PY{p}{,} \PY{l+s+s2}{\PYZdq{}}\PY{l+s+s2}{mV}\PY{l+s+s2}{\PYZdq{}}\PY{p}{)}
\PY{n+nb}{print}\PY{p}{(}\PY{l+s+s2}{\PYZdq{}}\PY{l+s+s2}{wconde =}\PY{l+s+s2}{\PYZdq{}}\PY{p}{,} \PY{n}{res}\PY{o}{.}\PY{n}{x}\PY{p}{[}\PY{l+m+mi}{1}\PY{p}{]}\PY{p}{,} \PY{l+s+s2}{\PYZdq{}}\PY{l+s+s2}{uS}\PY{l+s+s2}{\PYZdq{}}\PY{p}{)}
\PY{n+nb}{print}\PY{p}{(}\PY{l+s+s2}{\PYZdq{}}\PY{l+s+s2}{wcondi =}\PY{l+s+s2}{\PYZdq{}}\PY{p}{,} \PY{n}{res}\PY{o}{.}\PY{n}{x}\PY{p}{[}\PY{l+m+mi}{2}\PY{p}{]}\PY{p}{,} \PY{l+s+s2}{\PYZdq{}}\PY{l+s+s2}{uS}\PY{l+s+s2}{\PYZdq{}}\PY{p}{)}
\PY{n+nb}{print}\PY{p}{(}\PY{l+s+s2}{\PYZdq{}}\PY{l+s+s2}{gl =}\PY{l+s+s2}{\PYZdq{}}\PY{p}{,} \PY{n}{res}\PY{o}{.}\PY{n}{x}\PY{p}{[}\PY{l+m+mi}{3}\PY{p}{]}\PY{p}{,} \PY{l+s+s2}{\PYZdq{}}\PY{l+s+s2}{uS}\PY{l+s+s2}{\PYZdq{}}\PY{p}{)}
\PY{n+nb}{print}\PY{p}{(}\PY{l+s+s2}{\PYZdq{}}\PY{l+s+s2}{tausyne =}\PY{l+s+s2}{\PYZdq{}}\PY{p}{,} \PY{n}{res}\PY{o}{.}\PY{n}{x}\PY{p}{[}\PY{l+m+mi}{4}\PY{p}{]}\PY{p}{,} \PY{l+s+s2}{\PYZdq{}}\PY{l+s+s2}{ms}\PY{l+s+s2}{\PYZdq{}}\PY{p}{)}
\PY{n+nb}{print}\PY{p}{(}\PY{l+s+s2}{\PYZdq{}}\PY{l+s+s2}{tausyni =}\PY{l+s+s2}{\PYZdq{}}\PY{p}{,} \PY{n}{res}\PY{o}{.}\PY{n}{x}\PY{p}{[}\PY{l+m+mi}{5}\PY{p}{]}\PY{p}{,} \PY{l+s+s2}{\PYZdq{}}\PY{l+s+s2}{ms}\PY{l+s+s2}{\PYZdq{}}\PY{p}{)}
\end{Verbatim}
\end{tcolorbox}

    \begin{Verbatim}[commandchars=\\\{\}]
The optimazed parameters are:
Vrest = -70.00636190747738 mV
wconde = 0.09935438216276905 uS
wcondi = 1.087759474506438 uS
gl = 3.925898492990798 uS
tausyne = 0.009084476099794302 ms
tausyni = 0.06283249385911087 ms
    \end{Verbatim}

    \subsection*{d)}\label{d}

We plot the simulation with the optimized parameters and the given one.

    \begin{tcolorbox}[breakable, size=fbox, boxrule=1pt, pad at break*=1mm,colback=cellbackground, colframe=cellborder]
\prompt{In}{incolor}{16}{\boxspacing}
\begin{Verbatim}[commandchars=\\\{\}]
\PY{n}{plot\PYZus{}comparison}\PY{p}{(}\PY{n}{run\PYZus{}network}\PY{p}{(}\PY{n}{res}\PY{o}{.}\PY{n}{x}\PY{p}{[}\PY{l+m+mi}{0}\PY{p}{]}\PY{p}{,} \PY{n}{cm}\PY{p}{,} \PY{n}{ereve}\PY{p}{,} \PY{n}{erevi}\PY{p}{,} \PY{n}{res}\PY{o}{.}\PY{n}{x}\PY{p}{[}\PY{l+m+mi}{1}\PY{p}{]}\PY{p}{,} \PY{n}{res}\PY{o}{.}\PY{n}{x}\PY{p}{[}\PY{l+m+mi}{2}\PY{p}{]}\PY{p}{,} \PY{n}{res}\PY{o}{.}\PY{n}{x}\PY{p}{[}\PY{l+m+mi}{3}\PY{p}{]}\PY{p}{,} \PY{n}{res}\PY{o}{.}\PY{n}{x}\PY{p}{[}\PY{l+m+mi}{4}\PY{p}{]}\PY{p}{,} \PY{n}{res}\PY{o}{.}\PY{n}{x}\PY{p}{[}\PY{l+m+mi}{5}\PY{p}{]}\PY{p}{)}\PY{p}{)}
\end{Verbatim}
\end{tcolorbox}

    \begin{center}
    \adjustimage{max size={0.9\linewidth}{0.9\paperheight}}{output_38_0.png}
    \end{center}
    { \hspace*{\fill} \\}
    
    The two curves are almost indistiguishable. We then plot the normalized
difference in percentage:

    \begin{tcolorbox}[breakable, size=fbox, boxrule=1pt, pad at break*=1mm,colback=cellbackground, colframe=cellborder]
\prompt{In}{incolor}{17}{\boxspacing}
\begin{Verbatim}[commandchars=\\\{\}]
\PY{n}{plt}\PY{o}{.}\PY{n}{plot}\PY{p}{(}\PY{n}{np}\PY{o}{.}\PY{n}{arange}\PY{p}{(}\PY{l+m+mi}{0}\PY{p}{,}\PY{n+nb}{len}\PY{p}{(}\PY{n}{v}\PY{p}{)}\PY{o}{*}\PY{n}{dt}\PY{p}{,}\PY{n}{dt}\PY{p}{)}\PY{p}{,} \PY{l+m+mi}{100}\PY{o}{*}\PY{p}{(}\PY{n}{run\PYZus{}network}\PY{p}{(}\PY{n}{res}\PY{o}{.}\PY{n}{x}\PY{p}{[}\PY{l+m+mi}{0}\PY{p}{]}\PY{p}{,} \PY{n}{cm}\PY{p}{,} \PY{n}{ereve}\PY{p}{,} \PY{n}{erevi}\PY{p}{,} \PY{n}{res}\PY{o}{.}\PY{n}{x}\PY{p}{[}\PY{l+m+mi}{1}\PY{p}{]}\PY{p}{,} \PY{n}{res}\PY{o}{.}\PY{n}{x}\PY{p}{[}\PY{l+m+mi}{2}\PY{p}{]}\PY{p}{,} \PY{n}{res}\PY{o}{.}\PY{n}{x}\PY{p}{[}\PY{l+m+mi}{3}\PY{p}{]}\PY{p}{,} \PY{n}{res}\PY{o}{.}\PY{n}{x}\PY{p}{[}\PY{l+m+mi}{4}\PY{p}{]}\PY{p}{,} \PY{n}{res}\PY{o}{.}\PY{n}{x}\PY{p}{[}\PY{l+m+mi}{5}\PY{p}{]}\PY{p}{)}\PY{o}{\PYZhy{}}\PY{n}{data}\PY{p}{)}\PY{o}{/}\PY{n}{data}\PY{p}{)}
\PY{n}{plt}\PY{o}{.}\PY{n}{xlabel}\PY{p}{(}\PY{l+s+s2}{\PYZdq{}}\PY{l+s+s2}{t }\PY{l+s+s2}{\PYZdq{}}\PY{p}{,} \PY{n}{fontsize}\PY{o}{=}\PY{l+m+mi}{16}\PY{p}{)}
\PY{n}{plt}\PY{o}{.}\PY{n}{ylabel}\PY{p}{(}\PY{l+s+sa}{r}\PY{l+s+s2}{\PYZdq{}}\PY{l+s+s2}{\PYZdl{}}\PY{l+s+s2}{\PYZbs{}}\PY{l+s+s2}{epsilon\PYZdl{}}\PY{l+s+s2}{\PYZdq{}}\PY{p}{,} \PY{n}{fontsize}\PY{o}{=}\PY{l+m+mi}{16}\PY{p}{)}
\PY{n}{plt}\PY{o}{.}\PY{n}{show}\PY{p}{(}\PY{p}{)}
\end{Verbatim}
\end{tcolorbox}

    \begin{center}
    \adjustimage{max size={0.9\linewidth}{0.9\paperheight}}{output_40_0.png}
    \end{center}
    { \hspace*{\fill} \\}
    

    % Add a bibliography block to the postdoc
    
    
    
\end{document}
